\documentclass[times]{sig-alternate}





% *** MATH PACKAGES ***
%\usepackage[cmex10]{amsmath}

% *** SPECIALIZED LIST PACKAGES ***
%
% \usepackage{algorithmic}
% algorithmic.sty was written by Peter Williams and Rogerio Brito.
% This package provides an algorithmic environment fo describing algorithms.
% You can use the algorithmic environment in-text or within a figure
% environment to provide for a floating algorithm. Do NOT use the algorithm
% floating environment provided by algorithm.sty (by the same authors) or
% algorithm2e.sty (by Christophe Fiorio) as IEEE does not use dedicated
% algorithm float types and packages that provide these will not provide
% correct IEEE style captions. The latest version and documentation of
% algorithmic.sty can be obtained at:
% http://www.ctan.org/tex-archive/macros/latex/contrib/algorithms/
% There is also a support site at:
% http://algorithms.berlios.de/index.html
% Also of interest may be the (relatively newer and more customizable)
% algorithmicx.sty package by Szasz Janos:
% \usepackage[noendl]{algorithmicx}
% \usepackage[noend]{algpseudocode}
% http://www.ctan.org/tex-archive/macros/latex/contrib/algorithmicx/
% \renewcommand{\algorithmiccomment}[1]{\hskip 3em// \textit{#1}}

\usepackage{listings}
\lstset{
	language=C,
	morekeywords={iterate, until, for, every},
	basicstyle=\small\rmfamily,
	keywordstyle=\bfseries,
	numbers=left,
	columns=fullflexible,
	showstringspaces=false,
	xleftmargin=1.8em,
	frame=lines,
}




% *** ALIGNMENT PACKAGES ***
%
\usepackage{array}
% Frank Mittelbach's and David Carlisle's array.sty patches and improves
% the standard LaTeX2e array and tabular environments to provide better
% appearance and additional user controls. As the default LaTeX2e table
% generation code is lacking to the point of almost being broken with
% respect to the quality of the end results, all users are strongly
% advised to use an enhanced (at the very least that provided by array.sty)
% set of table tools. array.sty is already installed on most systems. The
% latest version and documentation can be obtained at:
% http://www.ctan.org/tex-archive/macros/latex/required/tools/


%\usepackage{mdwmath}
%\usepackage{mdwtab}
% Also highly recommended is Mark Wooding's extremely powerful MDW tools,
% especially mdwmath.sty and mdwtab.sty which are used to format equations
% and tables, respectively. The MDWtools set is already installed on most
% LaTeX systems. The lastest version and documentation is available at:
% http://www.ctan.org/tex-archive/macros/latex/contrib/mdwtools/


% IEEEtran contains the IEEEeqnarray family of commands that can be used to
% generate multiline equations as well as matrices, tables, etc., of high
% quality.


%\usepackage{eqparbox}
% Also of notable interest is Scott Pakin's eqparbox package for creating
% (automatically sized) equal width boxes - aka "natural width parboxes".
% Available at:
% http://www.ctan.org/tex-archive/macros/latex/contrib/eqparbox/





% *** SUBFIGURE PACKAGES ***
%\usepackage[tight,footnotesize]{subfigure}
% subfigure.sty was written by Steven Douglas Cochran. This package makes it
% easy to put subfigures in your figures. e.g., "Figure 1a and 1b". For IEEE
% work, it is a good idea to load it with the tight package option to reduce
% the amount of white space around the subfigures. subfigure.sty is already
% installed on most LaTeX systems. The latest version and documentation can
% be obtained at:
% http://www.ctan.org/tex-archive/obsolete/macros/latex/contrib/subfigure/
% subfigure.sty has been superceeded by subfig.sty.



%\usepackage[caption=false]{caption}
%\usepackage[font=footnotesize]{subfig}
% subfig.sty, also written by Steven Douglas Cochran, is the modern
% replacement for subfigure.sty. However, subfig.sty requires and
% automatically loads Axel Sommerfeldt's caption.sty which will override
% IEEEtran.cls handling of captions and this will result in nonIEEE style
% figure/table captions. To prevent this problem, be sure and preload
% caption.sty with its "caption=false" package option. This is will preserve
% IEEEtran.cls handing of captions. Version 1.3 (2005/06/28) and later 
% (recommended due to many improvements over 1.2) of subfig.sty supports
% the caption=false option directly:
%\usepackage[caption=false,font=footnotesize]{subfig}
%
% The latest version and documentation can be obtained at:
% http://www.ctan.org/tex-archive/macros/latex/contrib/subfig/
% The latest version and documentation of caption.sty can be obtained at:
% http://www.ctan.org/tex-archive/macros/latex/contrib/caption/




% *** FLOAT PACKAGES ***
%
\usepackage{fixltx2e}
% fixltx2e, the successor to the earlier fix2col.sty, was written by
% Frank Mittelbach and David Carlisle. This package corrects a few problems
% in the LaTeX2e kernel, the most notable of which is that in current
% LaTeX2e releases, the ordering of single and double column floats is not
% guaranteed to be preserved. Thus, an unpatched LaTeX2e can allow a
% single column figure to be placed prior to an earlier double column
% figure. The latest version and documentation can be found at:
% http://www.ctan.org/tex-archive/macros/latex/base/



\usepackage{stfloats}
% stfloats.sty was written by Sigitas Tolusis. This package gives LaTeX2e
% the ability to do double column floats at the bottom of the page as well
% as the top. (e.g., "\begin{figure*}[!b]" is not normally possible in
% LaTeX2e). It also provides a command:
%\fnbelowfloat
% to enable the placement of footnotes below bottom floats (the standard
% LaTeX2e kernel puts them above bottom floats). This is an invasive package
% which rewrites many portions of the LaTeX2e float routines. It may not work
% with other packages that modify the LaTeX2e float routines. The latest
% version and documentation can be obtained at:
% http://www.ctan.org/tex-archive/macros/latex/contrib/sttools/
% Documentation is contained in the stfloats.sty comments as well as in the
% presfull.pdf file. Do not use the stfloats baselinefloat ability as IEEE
% does not allow \baselineskip to stretch. Authors submitting work to the
% IEEE should note that IEEE rarely uses double column equations and
% that authors should try to avoid such use. Do not be tempted to use the
% cuted.sty or midfloat.sty packages (also by Sigitas Tolusis) as IEEE does
% not format its papers in such ways.





\usepackage{url}

% correct bad hyphenation here
\hyphenation{op-tical net-works semi-conduc-tor}

% from SuperComputing Glasswing paper
\usepackage{epsfig}
\usepackage{amssymb}
\usepackage{amsmath}
\usepackage{amsfonts}
\usepackage{pslatex}
% \usepackage{comment}
\usepackage{verbatim}
\usepackage{dcolumn}
\usepackage[noadjust]{cite}
\usepackage{fancyhdr}
\setcounter{secnumdepth}{4}

\begin{document}
\conferenceinfo{WOODSTOCK}{'97 El Paso, Texas USA}
%
% paper title
% can use linebreaks \\ within to get better formatting as desired
\title{Detecting Overlapping Communities within Graphs using an Accelerated Markov
Chain Monte Carlo Algorithm}


% author names and affiliations
% use a multiple column layout for up to two different
% affiliations
\numberofauthors{3}
\author{
\alignauthor{Ismail El-Helw}
\affaddr{Department of Computer Science}\\
\affaddr{Vrije Universiteit Amsterdam}\\
\affaddr{The Netherlands}\\
\email{ielhelw@cs.vu.nl}
\alignauthor{Rutger Hofman}
\affaddr{Department of Computer Science}\\
\affaddr{Vrije Universiteit Amsterdam}\\
\affaddr{The Netherlands}\\
\email{ielhelw@cs.vu.nl}
\alignauthor{Henri E. Bal}
\affaddr{Department of Computer Science}\\
\affaddr{Vrije Universiteit Amsterdam}\\
\affaddr{The Netherlands}\\
\email{ielhelw@cs.vu.nl}
}

% conference papers do not typically use \thanks and this command
% is locked out in conference mode. If really needed, such as for
% the acknowledgment of grants, issue a \IEEEoverridecommandlockouts
% after \documentclass

% make the title area
\maketitle

\begin{abstract}
The abstract goes here. DO NOT USE SPECIAL CHARACTERS, SYMBOLS, OR MATH IN YOUR TITLE OR ABSTRACT.

\end{abstract}

\category{A.0}{Applications}{Combinatorial and Data Intensive Application; Performance Analysis; Computational Science and Engineering Applications; Algorithms for Accelerators and Heterogeneous Systems}

\section{Introduction}

The past decade has witnessed a tremendous increase in the applicability and
usefulness of artificial intelligence in our daily lives. Much of this activity
was fueled by the mainstream adoption of machine learning approaches as they
provide tools to solve inherently difficult problems.
%
However, many of these techniques require a huge amount of computation
which severely limits the
scale of problems that can be tackled.

In this paper, we present the methodology followed in designing and
implementing a high-performance version of an existing Stochastic Gradient Markov Chain
Monte Carlo (SG-MCMC) machine learning algorithm that detects overlapping
communities in graphs. The algorithm analyzes pair-wise interactions between
entities in order to discover hidden attributes.
%
For instance, consider a social network represented as an undirected graph
where the vertices represent individuals and edges represent relations between
them. Given the relation information, the algorithm can identify latent groups
of individuals that represent shared interests.
%
This problem structure differs from graph partitioning or clustering as there
is a many-to-many relationship between individuals and interests. For example,
each individual can have multiple interests. Simultaneously, each interest group can
span multiple individuals.
%
Formally, this problem domain is known as Mixed-Membership Stochastic
Blockmodels (MMSB). The theory behind the algorithm is discussed in more detail
in~\cite{LiAW15}.

The focus of this work is on the computational efficiency and parallel
performance of the SG-MCMC algorithm. More specifically, we discuss the process
of accelerating the algorithm by developing aggressive optimizations targeting
multi-core CPUs and GPUs. The parallel algorithm achieves speedup factors up to 86, compared to a
well-tuned sequential C++ program which itself is a factor~1000-1500 faster than
the original Python/Numpy program developed by the algorithm's authors. From a
computational point of view, this algorithm differs from widespread machine
learning algorithms in several ways. First, it is highly data-intensive which
makes GPU acceleration particularly challenging. Second, owing
to the algorithm's stochastic nature, the majority of its memory access
patterns and data dependencies are non-deterministic. As a result,
straightforward
optimization attempts of the memory access patterns either fail or lead
to non-intuitive results.

Through careful analysis of the computation and data structures we show that
the algorithm's full state can be reduced by roughly 75\%. Compressing the
state significantly reduces the data intensity and allows for tackling larger
problems while maintaining all state in memory.
%
Further, by cataloguing and accounting for the various load and store
operations, we identified the highest priority locations of data reuse. In
order to circumvent the unclear optimization landscape, we developed an
effective kernel code generation mechanism that explores the benefits of
exploiting all permutations of the available optimization opportunities. These
optimizations include caching in shared memory, caching in the register file,
loop unrolling and explicit vectorization.

In summary, the contributions of this
work are:
\begin{itemize}
  \item Decreasing the algorithm's data intensity by eliminating 75\% of its
    memory footprint
  \item Tuning the algorithm's performance by maximizing data reuse and
    identifying the highest performing combination of optimizations through
    dynamic kernel code generation
  \item Achieving speedup factors of 21 and 86 over an optimized sequential
    program using a multi-core CPU and a GPU respectively
  \item A comparative performance analysis of the accelerated algorithm
    versions on a multi-core CPU and four GPUs, highlighting the particular
    optimization combinations that were successful per device.
\end{itemize}

The remainder of this paper is organized as follows. A description of the
sequential version of the algorithm and its data structures is provided in
Section~2. Section~3 delves into the design of the parallel algorithm and
discusses its contributions.  Section~4 provides an empirical evaluation of the
contributions of this work.  Section~5 presents an overview of related works.
Finally, Section~6 concludes.


\section{Related Work}

In this section, we discuss related machine learning projects that use
accelerators.
Immediately when GPUs evolved into generally programmable accelerators, machine
learners recognized the potential of GPUs for their science.
Much research
has been devoted to deep-learning implementations on GPUs. However, Mixed
Membership deduction (MMSB) is usually considered untractable for deep-learning
algorithms. In contrast, our algorithm belongs to the family of Bayesian
(approximative, stochastic) methods, where efficient algorithms for large-scale
data have recently been developed. A broad survey of Bayesian methods is given
in~\cite{DBLP:journals/corr/Zhu0H14}. Broadly speaking, Bayesian approximative
algorithms are subdivided into variational and Monte Carlo methods; Markov
Chain Monte Carlo (MCMC) is a subgroup of the latter. GPUs have been put to
use in MCMC algorithms, e.g.\ \cite{journals/bioinformatics/MedlarGSBK13}
uses GPUs to analyse parental linkage patterns in a biology context and
\cite{DBLP:journals/csda/WhiteP14} does the same to model terrorist activity.
Variational inference on MMSB is described
in~\cite{DBLP:conf/nips/GopalanMGFB12}, but they don't use GPUs. Latent
Dirichlet Allocation, another variety of Bayesian Approximation, is used on
GPUs by~\cite{DBLP:conf/nips/YanXQ09}.

Still, for high-dimension problems, the standard MCMC algorithms fall short in
performance. Recently, varieties of MCMC algorithms have been developed where
gradient information is used to speed up convergence. Langevin and Hamiltonian
dynamics are representatives of these
varieties~\cite{Girolami_riemannmanifold}.  Our algorithm uses Riemann Manifold
Langevin dynamics. In \cite{beam2014fast}, GPUs are used to perform Hamiltonian
descent, using Python interfaces to access the standard cuBLAS
library~\cite{cuBLAS}. They limit their optimizations for the GPU to
saving on data transfers between host and device memory. They do not apply
their framework to MMSB.

A different MMSB algorithm with stochastic gradient descent on the GPU is
the Online Tensor approach~\cite{DBLP:journals/corr/HuangNHVA13}. They deduce
overlapping communities for many of the same datasets as this paper, and they
report fast convergence. Their implementation uses the cuBLAS
library, and, unlike our work, there is no attempt to hand-optimize the GPU
kernels. Additionally, the algorithmic approach to solve the problem is
fundamentally different.
Since they target GPUs only, the datasets they
can handle are limited by the device memory of the GPU. Our implementation converges
quickly on the GPU but can also function with reduced speedup on a multicore
CPU. In the latter case, the dataset
size is only limited by the memory of the host machine.

\begin{comment}
Since our SG-MCMC algorithm is completely different from deep-learning
algorithms, it makes no real sense to compare to the learning frameworks
Theano, Caffe, cuDNN, or NVidia Digits, even though they target GPUs.

SG-MCMC work NOT on GPUs:

GPU work:
MCMC Hamiltonian:
	1. Andrew L. Beam, Sujit K. Ghosh, Jon Doyle
		Fast Hamiltonian Monte Carlo Using GPU Computing
		http://arxiv.org/pdf/1402.4089.pdf
MMSB tensor:

MCMC (not SG):
    1. Alan Medlar, Dorota Głowacka, Horia Stanescu, Kevin Bryson, Robert Kleta
       SwiftLink: Parallel MCMC linkage analysis utilising multicore CPU and GPU
       http://bioinformatics.oxfordjournals.org/content/early/2012/12/13/bioinformatics.bts704.full.pdf
    2. Marc Suchard, Chris Holmes, Mike West
       Some of the What?, Why?, How?, Who?  and Where?  of Graphics Processing Unit Computing for Bayesian Analysis
       https://stat.duke.edu/gpustatsci/GPU-ISBABull2010.pdf
\end{comment}

\begin{comment}
In this paper, we describe our custom RDMA D-KV (Distributed Key-Value)
store. Current RMDA D-KV store implementations are RamCloud~\cite{RamCloud},
Pilaf~\cite{Pilaf}, Herd~\cite{Herd} and FaRM~\cite{FaRM}. All these systems
use RDMA to implement a D-KV store. However, all of them are far more powerful
than our custom implementation -- and this power comes at a cost that we
can avoid. They implement a generic D-KV store that controls concurrency,
supports dynamic inserts and deletes, supports variable-sized values
(whose size may change at an update), and keys of arbitrary type. Because
of the nature of our distributed algorithm, we have to deal with none of
these issues. For us, values are fixed-size, allocated only at the initial
population, and remain alive forever. We have no concurrency between writes
and reads or other writes. Our keys are a contiguous range of integers. All
these properties together allow an extremely low-overhead implementation
that does not involve the remote host in any transaction.
\end{comment}

\section{SG-MCMC Algorithm Overview} 
\label{sec-algorithm}

In this section we describe the computational aspects of the SG-MCMC MMSB
algorithm. Moreover, we will introduce the data structures and notation that
will be used throughout this paper.
%
Kindly refer to~\cite{LiAW15} for a detailed explanation of
the statistical concepts of the algorithm.

The network graph $G$ consists of the undirected edges~$E$ and has $N$ vertices.
The algorithm starts by partitioning $G$ into the \textit{training
set}, the \textit{validation set} $H$ and the test set (the latter is not used
in our implementations). $H$ and the test set
are much smaller than $G$, typically between 1\% and 10\% of the edges
in~$G$.
The number of communities $K$
is specified as a model parameter to the algorithm.

The algorithm progresses by iteratively improving the global state of the learning
problem, using the training set. There are two pairs of data structures that
hold the global state. $\theta$, a $K{\times}2$ matrix, is used to
calculate
the community strength. The community strenghth represents the probability that two members in a community share
an edge. The actual community strength
is the vector $\beta$ of length $K$. As is shown in Equation~\ref{eqn-beta}, it is the
normalized version of~$\theta_{k,2}$.
The $\pi$ matrix of dimensions $N{\times}K$ represents the probability for
each vertex in $G$ to be a member of each community. It is the normalized
equivalent of the $\phi$ matrix of dimensions $N{\times}K$, on which the
calculations occur. The
definitions of $\beta$ and $\pi$ are:
%
\begin{eqnarray}
    \label{eqn-beta}
    \beta_{k} = \frac {\theta_{k,2}} {\theta^{sum}_k} &
    	\textrm{where} & \theta^{sum}_k = \sum_{j=1}^{2} \theta_{kj} \\
    \label{eqn-pi}
    \pi_{ik} = \frac {\phi_{ik}} {\phi^{sum}_i} &
    	\textrm{where} & \phi^{sum}_i = \sum_{k=1}^{K} \phi_{ik}
\end{eqnarray}

Pseudo-code for the algorithm is presented in
Listing~\ref{algorithm-pseudo-code}.

An iteration in the algorithm consists of 6 compute stages.
We will highlight the data accessed in the stages as this
determines the opportunities for parallelism.

\begin{figure}[tb]
\begin{comment}
Stochastic: iteratively:
 - sample a mini-batch $M$ (from the graph, or not from G)
 - for each node in the mini-batch:
   sample $n$ random 'neighbors' (not from G)
   for each edge (node, neighbor), calculate its contribution to the gradient
   add noise to the gradient contributions to be able to jump out of local minima
 - sum the gradients for each node in the mini-batch
 - using the gradients, infer new value of $\phi$ for each node in the mini-batch
 - update $\pi$ accordingly
 - calculate the gradients for $\beta$ for the edges in the mini-batch, and
   update $\beta$ and $\theta$
 - every so many iterations:
    + verify the quality of $\pi$ and $\beta$ against the validation set; metric is
      perplexity
    + terminate when perplexity changes less than $\epsilon$
\end{comment}


% \begin{figure}[h]
\begin{lstlisting}[mathescape,caption=SG-MCMC MMSB algorithm,label=algorithm-pseudo-code]
iterate
   sample a mini-batch $M$
   // $M$ may be a subset of $G$, or disjoint with $G$
   for each vertex $i$ in $M$:
      sample $n$ random 'neighbors' (not from $G$)
      for each edge ($i$, neighbor)
         calculate its contribution to the gradient in $\phi$
      // to be able to jump out of local minima:
      add noise to the gradient contributions
      update $\phi[i]$ using stepsize $s$
   for each vertex $i$ in $M$:
      update $\pi[i]$ according to changed $\phi[i]$
   for the edges in $M$:
      calculate gradients in $\theta$ and update $\theta$
   for the edges in $M$:
      update $\beta$ accordingly
   every so many iterations:
      // metric is perplexity:
      verify the quality of $\pi$ and $\beta$ against the validation set
until perplexity change is less than $\epsilon$
\end{lstlisting}
% \end{figure}

\end{figure}

The first stage (line~2 in the listing) randomly draws a mini-batch $M$,
using the "stratified random node" strategy~\cite{LiAW15}.
In this strategy, a coin toss is used to decide between two sample types.
%
The first sample type
chooses one random vertex $i$ and selects all of its edges to constitute the
mini-batch $M$. This sample type is referred to as \textit{link edges}. On the other
hand, the second sample type randomly draws a vertex $i$ and generates
random edges of the form ($i$, $j$) such that the edges are not in $G$. This
sample type is referred to as \textit{nonlink edges}. The set of vertices that
constitute the edges of the mini-batch $M$ are represented by $m$.

In the second stage (line~4), for each vertex $i$ in~$m$, a neighbor set of
size $n$ is randomly sampled with edges of the form $(i,j)$.

Stage~3, \textit{update\_phi} (line~6-10), calculates a gradient vector
$\nabla\phi_i$ for each vertex $i$ in the mini-batch, by iterating over
the edges~${(i,j)}$ in $i$'s neighbor set; the data that is used is
$\pi_i$, $\pi_j$,
and~$\beta$. The gradient $\nabla\phi_i$ is used to update~$\phi_i$. Random
noise is added to the update to prevent the algorithm from getting confined
to a local minimum.
Stage 4 (line~11-13), \textit{update\_pi}, updates $\pi_i$ so it remains
the normalized version of~$\phi_i$.

Stage~5, \textit{update\_theta} (line~14),
uses $\beta$ and $\pi_a$, $\pi_b$ for the edges $(a,b)$ in the mini-batch~$M$ to
calculate a
gradient vector~$\nabla\theta$. $\theta$ is updated using $\nabla\theta$, again
with the addition of noise. Stage~6, \textit{update\_beta} (line~16),
recalculates $\beta$ as the normalized
version of~$\theta_{k,2}$.

At regular intervals, the algorithm's global state is assessed by
evaluating the perplexity over the edges in the validation set $H$. The
\textit{perplexity} is a metric that represents the quality of the algorithm solution at
a given point in time. It is used to detect the algorithm's
convergence.
The
perplexity, as defined in~\cite{LiAW15}, is the exponential
of the average over time of the negative
log-likelihood of meeting a link edge. In this 7th stage, $\beta$ is used,
as are $\pi_a$ and $\pi_b$ for each edge $(a,b)$ in~$H$.

The graph $G$ is queried for membership in the stages \textit{update\_phi},
\textit{update\_beta}, and \textit{perplexity}. The validation set~$H$ is
traversed in \textit{perplexity}.


% \subsection{System Design and Implementation}

Based on the efficient C++ baseline version, a version for GPU accelaration
was constructed.
This involved optimizations for both efficient resource utilization and
parallelization.
This section provides an overview of the system's evolution in incremental
phases, identifying the key contributions and differences between consecutive
states. This section concludes with an in-depth evaluation of the performance
of our various optimizations on a number of GPU
accelerators.

% \subsection{An Efficient Sequential Baseline Version}
\label{sec-design-sequential}

Initially, the algorithm was implemented in Python~\cite{LiAW15}, as is common
with machine learning researchers and practitioners. It
relied on numpy~\cite{numpy} to perform the numerical computations without the
overhead associated with Python's interpreter. However, the algorithm also
depended on native Python data structures such as sets, dictionaries and
lists, which are far less efficient.
This section presents how the Python implementation was
converted into an efficient sequential C++ application which serves as a
baseline for the parallel implementations.

Initially, the Python code base was ported into equivalent C++ code.
The program structure remained unchanged and the Python data structures were
simply substituted with their corresponding alternatives in C++.

The next step was to remove a number of inefficiencies in the C++
implementation. For example, one recurring idiom in the Python implementation
was an expression of the form $a^y b^{1-y}$, implemented with floating-point
exponentiation, where $y$ is a variable that can
only hold values of 0 or~1. Therefore, the expression is simply a choice
between $a$ for $y=1$ and $b$ for $y=0$. However, as the compiler cannot infer
the value of $y$, both expressions $a$ and $b$ were computed and their
resulting values were raised to the power $y$ and $1-y$ respectively.
We transformed such expressions into conditional expressions which
compute either the expression $a$ or $b$ and avoid exponentiation.
%
% Similarly, we transformed some expressions to a more efficient form.
% For example, we replaced expressions of the form $a^{0.5}\times{b}^{0.5}$
% with $sqrt(ab)$.
Other optimizations were loop strength
reduction and common subexpression lifting.

In summary, a multitude of localized optimizations were applied to achieve
an efficient sequential baseline version, which is 1000..1500 times faster
than the original Python code, see Section~\ref{sec-evaluation}.
All transformations were rigorously tested to ensure they do
not modify the behavior of the algorithm.

\begin{comment}
Further, we replaced calls to the system's random
functions with a custom implementation of the random generator
\textit{xorshift\_128}~\cite{Marsaglia:2003:XR}. This way, random calls no longer
involve system calls, so we can support easy and fast multi-threaded random
calls by providing each thread with its private, differently seeded, random
generator.
\end{comment}


% \section{Accelerating on one Many-core Machine}

\subsection{Design of GPU Optimizations}

% Accelerator friendliness
% The design of an accelerated version of the MCMC algorithm necessitated several
% crucial modifications to allow for efficient parallelization. This section
% discusses the key design contributions to attain optimized
% parallel versions.

Since the algorithm is memory-bound, many of the optimizations we implemented
are concerned with
memory usage. For starters, we describe our fast, memory-efficient hash table on the GPU,
taylored towards
the algorithm's usage patterns. Then some optimizations reduce the total memory
footprint of the algorithm. Other
optimizations explore the opportunities offered by the faster regions of the
GPU memory hierarchy: data can be put into GPU shared memory, into thread-local
memory, or in registers which requires a more complicated analysis. One
other optimization is exploitation of the vectorization possibilities.

The state for the combination of $\pi$ and $\phi$ is permanently
stored in GPU memory, because copying of the relevant portions of these
data structures to the GPU in each iteration is too costly compared to the amount
of computation. The graph is stored both at the host and in GPU memory. The
host draws the mini-batches, because that operation is cheap,
and the data required for a minibatch is small.

% \subsection*{Using OpenCL and CUDA}
%OpenCL/CUDA
\begin{comment}
\textit{\bf !!! Remove the history approach !!!}
Initially, the system was developed using OpenCL~\cite{opencl} in order to
accelerate the computations of the algorithm. OpenCL was chosen as it provides
a common abstraction for a variety of compute devices.
However, NVidia's OpenCL SDK limits the
total memory allocations within a context to 4GB which severely limits the
problem sizes that can be tackled on GPUs. Therefore, we migrated the system to
use the abstraction layer CLCudaAPI~\cite{claduc}
to support both OpenCL and CUDA~\cite{Nickolls:2008:SPP:1365490.1365500}
as back-ends.
%
CLCudaAPI
hides the difference between OpenCL and the CUDA driver API.
For example, the library provides abstractions for Device, Context, Queue and
Buffer. Additionally, the implementations use a combination of inline functions
and preprocessor macros to abstract away the differences between OpenCL's and
CUDA's primitive types, address spaces and functions.
\end{comment}
The GPU code and interfacing was developed with CLCudaAPI~\cite{claduc}, which
is an API that targets both CUDA for NVidia GPUs and OpenCL for other GPUs and
multicore CPUs. It was not feasible to directly use OpenCL~\cite{opencl} since
NVidia's OpenCL SDK limits the total memory allocations to 4GB, which is
significantly smaller than the memory available in our GPUs.

%%%%%%%%%%%%%%%%%%%%%%%%%%%%%%%%%%%%%%%%%
\subsubsection{Fast Lookup of Graph Edges}
% Set/Cuckoo Hash

The algorithm relies on a set data structure to store the edges of the graph.
This set is queried frequently with randomly generated edges to check for their
membership. In order to improve the performance of such lookups, a custom set
implementation was developed that restricts the set's features based on an
observation of its usage patterns.
The graph is immutable, hence its maximum
size can be specified at creation time and need not be adjustable or provide
thread-safety guarantees. The set data
structure is built as a variant of a cuckoo
hash~\cite{Pagh:2004:CH:1006424.1006426} where each element is a 64-bit
field containing a tuple of the two 32-bit vertices. The cuckoo hash
uses two hash functions to index their corresponding spaces. Additionally, the
cuckoo hash supports up to 4~keys per bucket. This hash design allows for a load
factor upwards of 90\%, a very economical memory usage.
%% FIXME: Talk lookup hit/miss with 4 keys per bucket?

%%%%%%%%%%%%%%%%%%%%%%%%%%%%%%%%%%%%%%%%%
\subsubsection{Parallelization \& Data Dependencies}
% Division into kernels
The original algorithm was structurally reorganized into 4 main sections, each with
one or more kernels, data dependencies and synchronization requirements.
%
First, the sampling of a mini-batch of edges is done on the host as it is a
cheap operation. The mini-batch sampling is followed by the neighbor sampling
kernel which generates uniformly random neighbors for each vertex in the
mini-batch. 
%
Second, the predominant kernel, \textit{update\_phi} is invoked to calculate the
gradients for each vertex $i$ in the mini-batch \Minibatch and update the value of $\phi_i
\mid i \in \Minibatch$. Next, the \textit{update\_pi} kernel is invoked to normalize the
individual $\phi_i$ and store the result in the corresponding~$\pi_i$.
%
Third, the kernels \textit{update\_theta} and \textit{update\_beta} are invoked to modify the
global parameters.
%
Finally, perplexity calculation is performed in a dedicated kernel.

%%%%%%%%%%%%%%%%%%%%%%%%%%%%%%%%%%%%%%%%%
\subsubsection{Memory Footprint Reduction}
% Memory footprint halved

A key structural change applied to the algorithm is the lossless compression of
its state. This modification enabled the algorithm to process larger data sets
while maintaining all program state in memory. Moreover, as the algorithm is
data-intensive, a reduction of the state is accompanied by a decrease in its
data intensity.
%
The data structures that occupy most memory are $\pi$ and~$\phi$, both of
which are matrices of dimension ${N}\times{K}$. However, storing both
is redundant as $\pi$ is a row-normalized copy of~$\phi$, see
Table~\ref{tbl-symbols} in Section~\ref{sec-background}. The full storage for
$\phi$ is discarded; $\phi_{ik}$ values are calculated as
$\pi_{ik}/\phi^{sum}_i$, which requires maintenance of a vector
$\phi^{sum}$ of size $N$.
Moreover, the calculation for $\phi_{ik}$ can be cached usefully.
%
The $\phi$ matrix is required in two kernels only, namely, \textit{update\_phi} and
\textit{update\_pi}. Both kernels access $\phi_i$ only for vertices~$i$ in the
mini-batch~\Minibatch, so for each iteration, the calculated values for $\phi$ are cached
in a smaller temporary matrix of size
$M\times{K}$.
%
% In this context, the initialization of the algorithm's state is performed by
% normalizing the full $\pi$ matrix in-place without the need for an additional
% matrix. Moreover, the normalization denominator of each row is cached in
% $\phi^{sum}_{i}$.
%
% Thus each read-only access to $\phi_{i,j}$ can be substituted with
% $\pi_{i,j}\phi_{i}^{sum}$. Given that the value of $\phi_{i,j}$ is consistently
% used in conjunction with $\pi_{i,j}$ and that $\phi_{i}^{sum}$ is read once and
% reused multiple times, this
This transformation trades memory storage and
bandwidth for a minimal computation overhead.
%
% On the other hand, the newly computed $\phi$ values can be stored in the
% temporary matrix and later be normalized into their corresponding rows of
% $\pi$.

Thus, the original memory requirement for $\phi$, ${N}\times{K}$, is
reduced to a vector $\phi_{sum}$ of length~$N$ and a comparatively small
$M\times{K}$ matrix. For sufficiently large~$K$, this transformation roughly
halves the memory footprint of the algorithm.

%%%%%%%%%%%%%%%%%%%%%%%%%%%%%%%%%%%%%%%%%
% Single floating point vs double
Another modification to the original algorithm is migrating all of its
computation from double to single precision floating-point, analogous to the change
for the baseline implementation in Section~\ref{sec-seq-performance}.
%
This change affects the algorithm in multiple ways.
It again halves the memory
footprint of the algorithm and frees registers.
Besides, on our GPUs it
reduces
the computation intensity besides the data intensity.

%Finally, the algorithm requires randomly generated values from 3 different
%distributions. 

% \subsubsection{Edge-centric and Vertex-centric kernels}
\subsubsection{Memory Hierarchy Optimizations}
\label{gpu-design}

The work decomposition of the kernels ensures that each thread performs
independent computation in order to avoid expensive synchronization operations.
Edge-centric kernels, which operate over mini-batch edges, perform computations
over every edge in parallel.  Similarly, vertex-centric kernels that operate
over vertices in the mini-batch exploit parallelism across the selected
vertices.
% vectorization
Additionally, the kernels were vectorized to decrease instruction overhead.

% The GPU implementation builds on top of the CPU work decomposition scheme.
% However, instead of having every thread perform independent computations, each
% block of threads shares the work associated with the single edge or vertex for
% edge-centric and vertex-centric kernels.

Since all of the kernels are data-bound, several memory organization strategies
of varying complexity were investigated to speed up the algorithm by
exploiting the GPU's memory hierarchy.
% Update_phi
As a case study, a discussion of the \textit{update\_phi} kernel is provided as it is
the predominant component of the algorithm.

The \textit{update\_phi} kernel operates over every vertex~$i$ in the mini-batch and
requires two temporary vectors of length~$K$ to perform its computation. For
each vertex, it iterates over the randomly generated neighbors and computes a
vector of probabilities $Probs_i$ of length~$K$. The individual $Probs_i$ vectors
of each ($i$, neighbor) tuple are then used to update the gradients vector
$Grads_i$ for each vertex~$i$. Finally, the $\phi_i$ row is updated to
reflect the changes that were accumulated in $Grads_i$ for each vertex $i$ in the
mini-batch~\Minibatch.

A deeper analysis of the memory access patterns of the \textit{update\_phi} kernel
revealed the frequency and modality of access to the data structures.
%
The read-only accesses of $\pi_j$ of the randomly generated neighbors are unique
with a high probability. More precisely, each $vertex_i$ in the mini-batch
randomly samples a set of neighbor identifiers from the uniform distribution.
Given that the total number of sampled neighbors is much smaller than $N$,
there is a low likelihood of having duplicate samples which eliminates the
potential of data reuse. Therefore, these accesses provide limited or no
opportunities for optimization without interfering with the algorithm's
entropy.
%
The data structure usage patterns that are deterministic and most frequently
accessed in read/write mode are $Probs$ and $Grads$ respectively. Similarly,
$\pi_i$ for each vertex in the mini-batch is read repeatedly for the
calculation of $Probs$ per neighbor and again for the updates of $Grads$.

The following strategies present alternative methods of handling the
deterministic memory usage patterns of $Probs_i$, $Grads_i$ and $\pi_i$ for each
vertex in the mini-batch.

% Naive strategy
\paragraph*{\textbf{Naive strategy}} allocate temporary vectors in thread
local memory to store $Probs$ and $Grads$ which physically reside in
device memory. All memory accesses are coalesced to achieve the highest
possible bandwidth to and from device memory.

% Shared memory caching strategy
\paragraph*{\textbf{Shared memory strategy}} allocate the temporary vectors
$Probs_i$ and $Grads_i$
in shared memory. Furthermore, this strategy copies the $\pi_i$ of the selected
mini-batch vertex to shared memory in order to avoid repeated reads of the same
memory locations in device memory

% Register and shared memory using code generation and loop unrolling 
\paragraph*{\textbf{Code generation strategy}} dynamically generate the code of the kernel to
custom tailor its properties. For example, this strategy can control whether a
vector should be placed or cached in shared memory. Additionally, it can
control which vectors explicitly reside in registers by allocating space on the
stack frame, unrolling all inner loops of the kernel and substituting all
vector addressing with static values. The code generation strategy allows this
flexibility for the three vectors of concern, namely, $Probs_i$, $Grads_i$
and~$\pi_i$.
Hence, this strategy allows for 8 possible configurations denoted by
three letters, each of which is a choice between $R$ or $S$ to represent
$Register$ or $Shared$ respectively. For example, $SSR$ denotes that $Probs_i$,
$Grads_i$ and $\pi_i$ are placed in $Shared$, $Shared$ and $Register$
respectively.

% vectorization
Additionally, we investigated the use of vector data types to decrease the
instruction overhead and increase memory bandwidth for all strategies.


\subsection{Evaluation}
\label{sec-evaluation}

The following subsections discuss the performance evaluation of the algorithm
across its different stages. As the system underwent multiple design mutations,
a careful analysis is needed to identify the contributions of each change
independently.
%
First, an examination is provided detailing the construction of an efficient
sequential baseline version.
This spans
the code conversion from Python to C++ as well as the effects of applying
the localized optimizations discussed in
Section~\ref{sec-design-sequential}.
%
Second, we explore the performance benefits of parallelizing the computations
on a multi-core CPU using OpenCL.
%
Next, we assess the trade-offs associated with each of the GPU optimization
strategies and their performance effects on four GPUs spanning two chip
architecture generations.
%
Finally, we present the overall outcome of accelerating the algorithm by
displaying its fast convergence over large networks.

All experiments were conducted on the VU Amsterdam DAS5 cluster. The cluster
consists of 68 compute nodes each equipped with a dual 8-core Intel Xeon
E5-2630v3 CPU clocked at 2.40GHz, 64GB of memory and 8TB of storage.
Additionally, the cluster is fitted with a number of Nvidia GPUs including
GTX~\mbox{Titan-X}, GTX980, K40c and K20m; see Table~\ref{table-gpus} for an overview of
their properties. Table~\ref{table-snap} outlines the graph
properties of the networks used to evaluate the algorithm's performance. These
networks were obtained from the SNAP collection~\cite{snapnets}.

\begin{comment}
\begin{table}[tb]
\center\begin{tabular}{l r r c}
Network & \#Vertices & \#Edges \\
\hline
CA-HepPh        &    12,008 &    118,521 \\
com-DBLP        &   317,080 &  1,049,866 \\
% com-Youtube     & 1,134,890 &  2,987,624 \\
com-LiveJournal & 3,997,962 & 34,681,189 \\
\hline
\\[-1ex]
\end{tabular}
\caption{Network graphs from the Stanford Snap collection}
\label{table-snap}
\end{table}
\end{comment}

\begin{table}
  \centering
  \def\tabcolsep{0.2em}
  \begin{tabular}{l r r r p{9em}}
    Name            & \#Vertices &       \#Edges & \multicolumn{1}{c}{\#Ground-} & Description \\
                    &            &               & \multicolumn{1}{c}{truths}    &             \\
    \hline
    CA-HepPh        &    12,008  &    118,521    & \multicolumn{1}{c}{n/a}       & Paper citation network \\
    com-LiveJournal &  3,997,962 &    34,681,189 & 287,512        & Online blogging social network \\
    com-Friendster  & 65,608,366 & 1,806,067,135 & 957,154        & Online gaming social network \\
    com-Orkut       &  3,072,441 &   117,185,083 & 6,288,363      & Online social network \\
    com-Youtube     &  1,134,890 &     2,987,624 & 8,385          & Video-sharing social network \\
    com-DBLP        &    317,080 &     1,049,866 & 13,477         & Computer science bibliography network \\
    com-Amazon      &    334,863 &       925,872 & 75,149         & Product copurchasing network \\
    \hline
    \\[-1ex]
  \end{tabular}
  \caption{Summary of SNAP graph data sets used for evaluation.}
  \label{table-snap}
\end{table}


\subsubsection{Achieving an Efficient Baseline Sequential Version}

This subsection presents the results of transforming the algorithm implementation
from Python to C++, and then removing obvious inefficiencies, to achieve a
good baseline implementation for comparison with the parallel implementation.
%
We also investigate the performance impact of using
32-bit floating point numbers in stead of 64-bit doubles.
This hardly reduces the computation intensity for the Xeon, but it very much
affects data intensity because the data sizes are halved.
%
% It has been previously shown that stochastic learning
% algorithms do not require high precision in the presence
% of statistical approximations and the addition of random
% noise [?].

Due to the performance limitations of the Python implementation, the
experimental configuration of this section is minimal.
%
Namely, we use the CA-HepPh dataset; the number of
iterations is 1000.
%Note that these trial runs stop very far before convergence; using a small
%number of
%iterations is valid because the time spent in an iteration does not vary
%during the lifetime of the algorithm.
The selected number of communities is~1024, the mini-batch size~32, the
neighbor sample size~32.

The mini-batch sampling as described in Section~\ref{sec-background} randomly
chooses either a mini-batch of link edges, whose size is the degree of one
randomly selected vertex, or a mini-batch of nonlink edges, whose size is
specified as a model parameter. In this evaluation, we only select batches
of nonlink edges because that makes the mini-batch size, and hence the
execution times, deterministic. We separately validated that the time spent
\textit{per mini-batch vertex} for samples of link edges and nonlink edges
is fully consistent.

\begin{table}[b]
\center\begin{tabular}{l d{5.1} d{2.1} d{2.1} d{2.1}}
	&	& \multicolumn{3}{c}{C++} \\
\cline{3-5}
	& & 
            \multicolumn{1}{c}{Python} &
                \multicolumn{1}{c}{float64} &
		    \multicolumn{1}{c}{float32} \\
\multicolumn{1}{l}{Algorithm stage} &\multicolumn{1}{c}{Python} &
            \multicolumn{1}{c}{idiom} &
                \multicolumn{1}{c}{(baseline)} &
		    \multicolumn{1}{c}{baseline} \\
\hline
sample mini-batch       &      0.03 &  0.03 &  0.03 & 0.02 \\
sample neighbor sets    &      8.5  &  0.5  &  0.4  & 0.35 \\
update\_phi             &  9,014    & 52.7  &  8.9  & 4.9  \\
update\_pi              &     93.1  & 40.3  &  0.11 & 0.05 \\
update\_theta\_beta     &    339    &  2.1  &  0.76 & 0.55 \\
perplexity*             &    178    &  0.18 &  0.18 & 0.26 \\
\hline
\\[-1ex]
\end{tabular}
\caption{Performance comparison between Python, C++ that follows the Python
idioms, and our baseline C++ implementations. 1000 iterations, times in
\textrm{ms} per iteration; *perplexity time divided by 100 iterations.}
\label{table-python-comparison}
\end{table}

Table~\ref{table-python-comparison} compares performance between Python,
the C++ version labeled 'Python Idiom' with the inefficiencies inherited from
Python, then our baseline C++ version which has the inefficiencies removed,
and finally the same with 32-bit floats.
%
We present timings for the compute stages of the algorithm described in
Section~\ref{sec-background}, with the exception of
\textit{update\_theta\_beta} which combines \textit{update\_theta}
and \textit{update\_beta}.
%
The numbers are times in ms. For all stages except perplexity, the times
are per iteration.
%
Once every so many iterations (hundreds or thousands in production runs),
the perplexity calculation is invoked. The perplexity time in the table
has been divided by 100 iterations; per perplexity invocation, it is large in
comparison with the iteration stage times, but it is amortized over those
iterations.

In all implementations, \textit{update\_phi} dominates the computation.
The translation from Python into 'Python Idiom' C++ speeds up this stage
by a factor~171, even though the Python implementation uses numpy for its data
structures. Comparison of the C++ table entries for 'Python idiom' and
64-bit float baseline shows that our C++ optimizations speed up this stage
by another factor of~6. Reducing the floating-point precision to 32~bits
doubles this factor, which shows that the algorithm is very much data-dominant.
%
\textit{update\_pi} has a disproportionately large speed
gain after removing inefficiencies. This was attained by limiting the
update to $\pi$ to only those values of $\pi$ that have changed in this
iteration, in stead of the full $\pi$ array (which, incidentally, is handled
rather efficiently by numpy).
\textit{update\_theta\_beta} gave fewer opportunities for C++ optimization,
in the sampling stages there was none. In \textit{mini-batch-sampling} there
is no speedup compared to Python because it consists of a call to
a fast Python random primitive that cannot be bettered from C++.
The total performance gain between Python and the 64-bit float baseline C++
implementation exceeds a factor of~1000, and this grows to a factor of over~1500
if the floating-point precision is reduced to 32~bits. The 32-bit precision
implementation will be the baseline for the parallel performance comparison.

\begin{comment}
The introduction of a custom user-space random generator brings at most
a very small
benefit. We show it, because it is necessary for the multi-threaded
implementations described in the next section, and this measurement serves to
prove that it does not harm execution speed.
\end{comment}

\subsection{Analysis of CPU Parallelism}

\begin{comment}
  \begin{table}[t]
    \centering
    \begin{tabular}{l l c c c c}
      Comparison & Data Set        & K    & M    & n  & Iterations \\
      \hline
      Python/Baseline    & CA-HepPh        & 1024 & 32   & 32 & 1000 \\
      Parallel CPU/GPU   & com-DBLP        & 1024 & 4096 & 32 & 1000 \\
      Parallel CPU       & com-LiveJournal & 1024 & 4096 & 32 & 1000 \\
    \end{tabular}
    \caption{Experiment parameter reference.}
    \label{exp-params}
  \end{table}
\end{comment}

\begin{table}[b]	% [htb]
  \centering
  \begin{tabular}{l d{2.8} d{2.8}}
    Kernel & Time (seconds) \\
    \hline
    PPX CALC    &  0.0364737 \\
    PPX ACCUM   &  0.083 \\
    SAMPLING    &  0.535599 \\
    UPDATE\_PHI & 25.6598 \\
    UPDATE\_PI  &  0.645875 \\
    THETA SUM   &  0.0483902 \\
    GRADS PAR   &  1.92919 \\
    GRADS SUM   &  9.31122 \\
    UPDATE THETA&  0.0548013 \\
    NORM THETA  &  0.001 \\
    \hline
    TOTAL    & 38.5858 \\
  \end{tabular}
  \caption{Performance break-down of multi-core CPU version without
  vectorization. Model parameters: dataset com-DBLP; K=1024, m=4096, n=32.}
  \label{TABLE-CPU}
\end{table}

This section discusses the use of the multi-core CPU available on the DAS5
cluster. The parallel OpenCL version divides the work across the CPU cores and
performs independent calculations concurrently. As shown in
Table~\ref{TABLE-CPU}, the dominant kernel in the computation is
\textit{update\_phi},
which accounts for 66.5\% of the computation time. Without exploiting
the dual 8-core processor's vectorization capabilities, the speedup relative to
the baseline sequential C++ version is~9.8.

% CPU
\begin{figure}[htb]
\centering
\epsfig{file=plots/cpu-vector.eps, width=\columnwidth}
\caption{Performance of CPU for varying vector width.}
\label{FIG-CPU-VECTOR}
\end{figure}

In addition to applying computations in parallel, we investigated the use of
the CPU's SIMD instructions to maximize its resource utilization.
Figure~\ref{FIG-CPU-VECTOR} presents the performance obtained by vectorizing
the kernels with varying vector widths. A key aspect in this figure is the
diminishing performance benefit for higher vector widths. As the computational
performance increases, the memory throughput becomes the leading performance
bottleneck. Moreover, using 16-wide SIMD instructions gave a slight performance
penalty compared to 8-wide SIMD. The 8-wide vector version improves the speedup
relative to the baseline version from 9.8 to~20.9.

\begin{table}[b]	% [htb]
% \center\begin{tabular}{l c c c c c c c c d{3.1}}
% Device & Architecture & cores & Clock & \multicolumn{2}{c}{GFlops} & L2 Cache & Memory & Bandwidth \\ 
%        &              &       & MHz   & single & double            & KB       & GB     & (GB/s) \\ 
% \hline
% K20m        & Tesla   &  2496 &  706        & 3520 & 1170 &      &  5 & 208 \\
% K40c        & Tesla   &  2880 &  745        & 4290 & 1430 &      & 12 & 208 \\
% GTX 980     & Maxwell &  2816 & 1126 (1216) & 4612 &  144 &      &  4 & 336.5 \\
% GTX Titan-X & Maxwell &  3072 & 1000 (1075) & 6144 &  192 & 2048 & 12 & 336.5 \\
%
%
%###########################################################################################
% SPECS FROM:
% http://www.geforce.com/hardware/desktop-gpus/geforce-gtx-titan-x/specifications
% http://www.geforce.com/hardware/desktop-gpus/geforce-gtx-980/specifications
% http://www.nvidia.com/content/PDF/kepler/Tesla-K40-Active-Board-Spec-BD-06949-001_v03.pdf
% http://www.nvidia.com/content/PDF/kepler/Tesla-K20-Active-BD-06499-001-v04.pdf
%
% FLOPS DATA:
% K40: http://international.download.nvidia.com/pdf/kepler/TeslaK80-datasheet.pdf
% K20 & K40: http://www.nvidia.com/content/tesla/pdf/nvidia-tesla-kepler-family-datasheet.pdf
% GTX980 & TITANX: https://en.wikipedia.org/wiki/List_of_Nvidia_graphics_processing_units
\center\begin{tabular}{lc c c c c}
                         && \multicolumn{4} {c}{Device} \\
\cline{3-6}
Specification            && K20m  & K40c  & GTX 980 & GTX Titan-X \\
\cline{1-1}\cline{3-6}
Number of Cores          && 2496  & 2880  & 2048    & 3072  \\
Clock (MHz)              && 706   & 745   & 1126    & 1000  \\
GFlops (single)          && 3520  & 4290  & 4612    & 6144  \\
GFlops (double)          && 1170  & 1430  & 144     & 192   \\
Memory (GB)              && 5     & 12    & 4       & 12    \\
Bandwidth (GB/s)         && 208   & 288   & 224     & 336.5 \\
\cline{1-1}\cline{3-6}
\end{tabular}
\caption{Properties of the GPUs used in the evaluation}
\label{table-gpus}
\end{table}

\subsection{Analysis of GPU Parallelism}
As discussed in Section~\ref{gpu-design}, there are 10 distinct flavors of the
GPU version of the algorithm. Specifically, there is the naive strategy, the
shared strategy and 8 variations of the code generation strategy. This
section investigates the effectiveness of each flavor on Nvidia's GTX TitanX,
GTX980, K40c and K20m.

% TITANX
\begin{figure*}[t]	% [htb]
  \centering
  \epsfig{file=plots/titanx-1024-strategies-w1.eps, width=\textwidth}
  \caption{Execution time of 1000 \textit{update\_phi} invocations using the TitanX GPU,
  without explicit kernel vectorization, across a sweep of
  \textit{update\_phi} thread block
  sizes. Relevant model parameters: K=1024, M=4096, n=32.}
  \label{titanx-w1-sweep}
\end{figure*}

% TITANX K=2K
\begin{figure*}[t]	% [htb]
  \centering
  \epsfig{file=plots/titanx-2048-strategies-w1.eps, width=\textwidth}
  \caption{Execution time of 1000 \textit{update\_phi} invocations using the TitanX GPU,
  without explicit kernel vectorization, across a sweep of
  \textit{update\_phi} thread block
  sizes. Relevant model parameters: K=2048, M=4096, n=32.}
  \label{titanx-w1-sweep-2k}
\end{figure*}

Figure \ref{titanx-w1-sweep}(a) presents the performance of the GTX TitanX GPU
for all 10 strategy flavors, without explicitly vectorizing the kernels. The
x-axis represents different \textit{update\_phi} thread block sizes while the y-axis presents
the total execution time of 1000 invocations of the \textit{update\_phi} kernel. The
naive and shared strategies are labeled NAIVE and SHARED respectively. Further,
each flavor of the code generation strategy is labeled by GEN followed by the 3
choices that identify it.
%
The wide performance range inhibits the readability of the figure,
therefore, a more focused Figure~\ref{titanx-w1-sweep}(b) is provided. As would
be expected, the naive strategy exhibits the worst performance over all
thread block
sizes, as it does not explicitly cache repeated device memory read operations.
The SHARED and \textit{GEN-SSS} strategies come next in terms of performance.
Both strategies cache $Grads_i$, $Probs_i$ and $\pi_i$ in shared memory but
differ in one aspect. Namely, \textit{GEN-SSS} explicitly unrolls the internal
loops of the kernel. However, there is no significant performance difference
between them. The other flavors of the code generation strategy attain higher
performance as they unroll internal loops as well as cache data in
registers. The TitanX obtains the best performance with the \textit{GEN-RSS}
strategy and a thread block size of 64. The results of explicitly vectorized kernels
are omitted as they obtain worse performance on the TitanX.

A key model parameter that affects the behavior of the optimization strategies
is the number of communities $K$. Figure~\ref{titanx-w1-sweep-2k} presents the
same model configuration as in Figure~\ref{titanx-w1-sweep} but $K=2048$
instead of $K=1024$. One key difference between the two figures is the optimal
thread block size. An increase in $K$ comes with a proportional increase in
the size of shared memory space that is used by each thread block for the
strategies that employ shared memory. Similarly, GEN strategies that use the
register file will require additional space. Therefore, the number of
concurrent thread blocks that can execute on a single streaming multiprocessor
will decrease, minimizing the GPU's occupancy and utilization. This effect is
most clear when comparing the NAIVE with the SHARED and GEN-SSS strategies. In
this case, the NAIVE strategy outperforms both SHARED and GEN-SSS for a
thread block
size of 32 due to their low occupancy. This limitation can be counteracted by
selecting a larger thread block size which in turn increases the computation
concurrency and occupancy.

% K40
\begin{figure*}[t]	% [hbt]
  \centering
  \epsfig{file=plots/k40-1024-strategies-w2.eps, width=\textwidth}
  \caption{Execution time of 1000 \textit{update\_phi} invocations using the K40c GPU, with
  explicit kernel vectorization of width 2, across a sweep of
  \textit{update\_phi} thread block
  sizes. Relevant model parameters: K=1024, M=4096, n=32.}
  \label{k40-w2-sweep}
\end{figure*}

% \looseness=-1
Figures~\ref{k40-w2-sweep} (a) and (b) present the performance of the K40c GPU
for the same experimental configuration as shown for the TitanX in
Figure~\ref{titanx-w1-sweep} with one exception. Namely, the code version used
for this plot is explicitly vectorized with a vector width of~2. The results
for the other code versions with vector width 4 and no vectorization are
omitted as they exhibit lower performance.
%
Surprisingly, Figure~\ref{k40-w2-sweep} shows that the NAIVE strategy
outperforms SHARED and some of the GEN strategies. This can be explained by the
unique properties of the Tesla Super Computing line of products to which the
K40c belongs. These GPUs include enhanced L2 caching mechanisms that
accelerate repeated and sparse memory accesses. This is especially advantageous
as it caches repeated reads across streaming multiprocessors. However, the
highest performance is attained by \textit{GEN-RRS} which explicitly employs
registers for both $Probs_i$ and $Grads_i$.

The performance results obtained from the GTX TitanX and Tesla K40c GPUs
reinforce the importance of customizing compute kernels to each GPU's specific
architecture and capabilities. For instance, each GPU achieved its highest
performance by employing a different strategy. Moreover, each GPU displayed
different strategy-performance orderings.

\begin{figure}[b]	% [htb]
  \centering
  \epsfig{file=plots/gpus.eps, width=0.9\columnwidth}
  \caption{Speedup comparison of best performing parallel configurations of
  each compute device normalized over baseline C++ version. Relevant model
  parameters: K=1024, M=4096, n=32.}
  \label{gpus-fig}
\end{figure}

\begin{figure}[b]	% [htb]
  \centering
  \epsfig{file=plots/gpu-vector-sweep.eps, width=0.9\columnwidth}
  \caption{Performance of the best performing strategy per GPU, varying the
  vector width, normalized over the non-vectorized kernel. Relevant model
  parameters: K=1024, M=4096, n=32.}
  \label{gpus-v-sweep}
\end{figure}

\subsection{Comparison of Compute Devices}
Figure~\ref{gpus-fig} compares the highest speedup achieved by each of the
GTX TitanX, GTX980, K40c, K20m and the Intel Xeon CPU relative to the baseline
sequential C++ version. These results are consistent with the relative
capabilities of each device as listed in Table~\ref{table-gpus}. For instance,
the TitanX achieves the highest speedup of 86 relative to the baseline. At the
other end, the parallel CPU version attains a speedup factor of~20.9.

As vectorization is a popular GPU optimization, it is important to note that
both GTX GPUs achieved their highest performance with non-vectorized kernels.
On the other hand, the Tesla GPUs obtained the best performance with a vector
width of~2. Figure~\ref{gpus-v-sweep} presents the execution time of the best
performing strategy for each of the 4 GPUs. In this figure, the performance is
normalized over the non-vectorized kernel version for each GPU. For instance,
the TitanX exhibits an overhead factor of approximately 1.8 when using a vector
width of~4. On the other hand, the K40c improves its performance by roughly
20\% when it uses a vector width of~2 compared to the non-vectorized kernel.
Therefore, explicit vectorization of the kernels can be either useful or
harmful depending on the GPU architecture and the specific problem it is
applied to.





\begin{comment}
\begin{figure}[tb]	% [htb]
  \centering
  \epsfig{file=plots/ppx-gpu.eps, width=\columnwidth}
  \caption{Perplexity convergence of com-DBLP on the Maxwell \mbox{Titan-X}~GPU. Number
  of communities chosen to maximally fill the GPU memory.}
  \label{fig-ppx-gpu}
\end{figure}

\begin{figure}[tb]	% [htb]
  \centering
  \epsfig{file=plots/ppx-cpu.eps, width=\columnwidth}
  \caption{Perplexity convergence of com-LiveJournal on the multi-core~CPU.
  Number of communities chosen to maximally fill the CPU memory.}
  \label{fig-ppx-cpu}
\end{figure}
\end{comment}

\begin{figure}[tb]	% [htb]
  \centering
  \epsfig{file=plots/ppx-gpu-cpu.eps, width=0.45\textwidth}
  \caption{Perplexity convergence. (a)~com-DBLP on the Maxwell \mbox{Titan-X}~GPU.
  (b)~com-LiveJournal on the multi-core~CPU. Number of communities chosen to
  maximally fill the memory of GPU and CPU respectively. M=4K, $|\Neighbors|$=32.}
  \label{fig-ppx-gpu-cpu}
\end{figure}

\subsubsection{Algorithm Convergence}

This section presents the evolution of the perplexity, i.e.\ the metric for
the solution quality, as a function of time for
two datasets, whose properties are given in Table~\ref{table-snap}. This
analysis shows that our implementations of the algorithm indeed achieve
convergence, and it gives an impression of the time required to achieve
convergence for the largest datasets and community sizes that fit into the memory
of a GPU and a CPU respectively. The dominant customer of memory is the
$\pi$ matrix, sized $N{\times}K$ for $K$ communities and $N$ vertices in the
graph. The runtime per iteration is independent of~$N$, but it depends on the
product of~$K$, the mini-batch size~$M$, and the neighbor sample size~$|\Neighbors|$. On
the other hand, the number of iterations required to achieve sufficient
convergence does grow with the size of the graph.

Figure~\ref{fig-ppx-gpu-cpu}(a) shows the perplexity development for the dataset
com-DBLP, with $N=317,080$. It is computed on the Maxwell \mbox{Titan-X} GPU with input
parameter set $K=4096$, $m=4096$, $n=32$. Convergence is reached after 6~hours.

Figure~\ref{fig-ppx-gpu-cpu}(b) shows the perplexity convergence for
com-LiveJournal, with $N=3,997,962$, computed on a multi-core DAS5 node.
To fill the CPU core memory~(64GB), the input parameter set is $K$=3072,
$M$=4096, $|\Neighbors|$=32. Sufficient convergence is reached after 18~hours. Note
that this network graph is so large that the GTX~\mbox{Titan-X} would be unable to
process more than $K$=256 communities.


% \section{Conclusion}
Modern machine learning algorithms are empowering their
users to solve or approximate solutions to previously intractable problems.
However, these advancements come at the cost of significant computational
complexity and long running learning processes which hinder our ability to reap
such algorithms' benefits when applied to large problems. There is a clear need
to assess such algorithms and identify optimization opportunities in order to
scale their performance.

The SG-MCMC algorithm discussed in this paper posed additional computational
challenges due to its unique stochastic nature and high data intensity. Unlike
common machine learning algorithms, its data dependencies and memory access
patterns are nondeterministic. In this paper, we presented our methodology of
improving the algorithm's performance by fundamentally restructuring it to
cater for concurrency.

We showed that the algorithm's state can be reduced by 75\% after a thorough
analysis of the computational patterns and data structures which significantly
reduces its data intensity. Further, we navigated the complex optimization
landscape by dynamically generating kernel code and testing different
combinations of optimizations. The outcome of these efforts culminated in
significant speedup factors of 21 and 86 using a multi-core CPU and a GPU
respectively. These speedup numbers where achieved in comparison to an already
optimized sequential implementation. Finally, we evaluated the performance of
the parallel algorithm across several GPUs, highlighting the difference between
their optimal configurations.

The outcome of this study reinforces the significance of avoiding premature
optimization as it can lead to unexpected results. In particular, the success
of common GPU optimizations depends on the particular device in use and the
problem it is applied to.

% use section* for acknowledgement
\section*{Acknowledgment}

The authors would like to thank...
more thanks here


% trigger a \newpage just before the given reference
% number - used to balance the columns on the last page
% adjust value as needed - may need to be readjusted if
% the document is modified later
%\IEEEtriggeratref{8}
% The "triggered" command can be changed if desired:
%\IEEEtriggercmd{\enlargethispage{-5in}}

% references section

% can use a bibliography generated by BibTeX as a .bbl file
% BibTeX documentation can be easily obtained at:
% http://www.ctan.org/tex-archive/biblio/bibtex/contrib/doc/
% The IEEEtran BibTeX style support page is at:
% http://www.michaelshell.org/tex/ieeetran/bibtex/
%\bibliographystyle{IEEEtran}
% argument is your BibTeX string definitions and bibliography database(s)
%\bibliography{IEEEabrv,../bib/paper}
%
% <OR> manually copy in the resultant .bbl file
% set second argument of \begin to the number of references
% (used to reserve space for the reference number labels box)
\begin{thebibliography}{1}

\bibitem{IEEEhowto:kopka}
H.~Kopka and P.~W. Daly, \emph{A Guide to \LaTeX}, 3rd~ed.\hskip 1em plus
  0.5em minus 0.4em\relax Harlow, England: Addison-Wesley, 1999.

\end{thebibliography}




% that's all folks
\end{document}


