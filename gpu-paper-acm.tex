\documentclass[times]{sig-alternate}





% *** MATH PACKAGES ***
%\usepackage[cmex10]{amsmath}

% *** SPECIALIZED LIST PACKAGES ***
%
% \usepackage{algorithmic}
% algorithmic.sty was written by Peter Williams and Rogerio Brito.
% This package provides an algorithmic environment fo describing algorithms.
% You can use the algorithmic environment in-text or within a figure
% environment to provide for a floating algorithm. Do NOT use the algorithm
% floating environment provided by algorithm.sty (by the same authors) or
% algorithm2e.sty (by Christophe Fiorio) as IEEE does not use dedicated
% algorithm float types and packages that provide these will not provide
% correct IEEE style captions. The latest version and documentation of
% algorithmic.sty can be obtained at:
% http://www.ctan.org/tex-archive/macros/latex/contrib/algorithms/
% There is also a support site at:
% http://algorithms.berlios.de/index.html
% Also of interest may be the (relatively newer and more customizable)
% algorithmicx.sty package by Szasz Janos:
% \usepackage[noendl]{algorithmicx}
% \usepackage[noend]{algpseudocode}
% http://www.ctan.org/tex-archive/macros/latex/contrib/algorithmicx/
% \renewcommand{\algorithmiccomment}[1]{\hskip 3em// \textit{#1}}

\usepackage{listings}
\lstset{
	language=C,
	morekeywords={iterate, until, for, every},
	basicstyle=\small\rmfamily,
	keywordstyle=\bfseries,
	numbers=left,
	columns=fullflexible,
	showstringspaces=false,
	xleftmargin=1.8em,
	frame=lines,
}




% *** ALIGNMENT PACKAGES ***
%
\usepackage{array}
% Frank Mittelbach's and David Carlisle's array.sty patches and improves
% the standard LaTeX2e array and tabular environments to provide better
% appearance and additional user controls. As the default LaTeX2e table
% generation code is lacking to the point of almost being broken with
% respect to the quality of the end results, all users are strongly
% advised to use an enhanced (at the very least that provided by array.sty)
% set of table tools. array.sty is already installed on most systems. The
% latest version and documentation can be obtained at:
% http://www.ctan.org/tex-archive/macros/latex/required/tools/


%\usepackage{mdwmath}
%\usepackage{mdwtab}
% Also highly recommended is Mark Wooding's extremely powerful MDW tools,
% especially mdwmath.sty and mdwtab.sty which are used to format equations
% and tables, respectively. The MDWtools set is already installed on most
% LaTeX systems. The lastest version and documentation is available at:
% http://www.ctan.org/tex-archive/macros/latex/contrib/mdwtools/


% IEEEtran contains the IEEEeqnarray family of commands that can be used to
% generate multiline equations as well as matrices, tables, etc., of high
% quality.


%\usepackage{eqparbox}
% Also of notable interest is Scott Pakin's eqparbox package for creating
% (automatically sized) equal width boxes - aka "natural width parboxes".
% Available at:
% http://www.ctan.org/tex-archive/macros/latex/contrib/eqparbox/





% *** SUBFIGURE PACKAGES ***
%\usepackage[tight,footnotesize]{subfigure}
% subfigure.sty was written by Steven Douglas Cochran. This package makes it
% easy to put subfigures in your figures. e.g., "Figure 1a and 1b". For IEEE
% work, it is a good idea to load it with the tight package option to reduce
% the amount of white space around the subfigures. subfigure.sty is already
% installed on most LaTeX systems. The latest version and documentation can
% be obtained at:
% http://www.ctan.org/tex-archive/obsolete/macros/latex/contrib/subfigure/
% subfigure.sty has been superceeded by subfig.sty.



%\usepackage[caption=false]{caption}
%\usepackage[font=footnotesize]{subfig}
% subfig.sty, also written by Steven Douglas Cochran, is the modern
% replacement for subfigure.sty. However, subfig.sty requires and
% automatically loads Axel Sommerfeldt's caption.sty which will override
% IEEEtran.cls handling of captions and this will result in nonIEEE style
% figure/table captions. To prevent this problem, be sure and preload
% caption.sty with its "caption=false" package option. This is will preserve
% IEEEtran.cls handing of captions. Version 1.3 (2005/06/28) and later 
% (recommended due to many improvements over 1.2) of subfig.sty supports
% the caption=false option directly:
%\usepackage[caption=false,font=footnotesize]{subfig}
%
% The latest version and documentation can be obtained at:
% http://www.ctan.org/tex-archive/macros/latex/contrib/subfig/
% The latest version and documentation of caption.sty can be obtained at:
% http://www.ctan.org/tex-archive/macros/latex/contrib/caption/




% *** FLOAT PACKAGES ***
%
\usepackage{fixltx2e}
% fixltx2e, the successor to the earlier fix2col.sty, was written by
% Frank Mittelbach and David Carlisle. This package corrects a few problems
% in the LaTeX2e kernel, the most notable of which is that in current
% LaTeX2e releases, the ordering of single and double column floats is not
% guaranteed to be preserved. Thus, an unpatched LaTeX2e can allow a
% single column figure to be placed prior to an earlier double column
% figure. The latest version and documentation can be found at:
% http://www.ctan.org/tex-archive/macros/latex/base/



\usepackage{stfloats}
% stfloats.sty was written by Sigitas Tolusis. This package gives LaTeX2e
% the ability to do double column floats at the bottom of the page as well
% as the top. (e.g., "\begin{figure*}[!b]" is not normally possible in
% LaTeX2e). It also provides a command:
%\fnbelowfloat
% to enable the placement of footnotes below bottom floats (the standard
% LaTeX2e kernel puts them above bottom floats). This is an invasive package
% which rewrites many portions of the LaTeX2e float routines. It may not work
% with other packages that modify the LaTeX2e float routines. The latest
% version and documentation can be obtained at:
% http://www.ctan.org/tex-archive/macros/latex/contrib/sttools/
% Documentation is contained in the stfloats.sty comments as well as in the
% presfull.pdf file. Do not use the stfloats baselinefloat ability as IEEE
% does not allow \baselineskip to stretch. Authors submitting work to the
% IEEE should note that IEEE rarely uses double column equations and
% that authors should try to avoid such use. Do not be tempted to use the
% cuted.sty or midfloat.sty packages (also by Sigitas Tolusis) as IEEE does
% not format its papers in such ways.





\usepackage{url}

% correct bad hyphenation here
\hyphenation{op-tical net-works semi-conduc-tor}

% from SuperComputing Glasswing paper
\usepackage{epsfig}
\usepackage{amssymb}
\usepackage{amsmath}
\usepackage{amsfonts}
\usepackage{pslatex}
% \usepackage{comment}
\usepackage{verbatim}
\usepackage{dcolumn}
\usepackage[noadjust]{cite}
\usepackage{fancyhdr}
\setcounter{secnumdepth}{4}

\begin{document}
\conferenceinfo{WOODSTOCK}{'97 El Paso, Texas USA}
%
% paper title
% can use linebreaks \\ within to get better formatting as desired
\title{Detecting Overlapping Communities within Graphs using an Accelerated Markov
Chain Monte Carlo Algorithm}


% author names and affiliations
% use a multiple column layout for up to two different
% affiliations
\numberofauthors{3}
\author{
\alignauthor{Ismail El-Helw}
\affaddr{Department of Computer Science}\\
\affaddr{Vrije Universiteit Amsterdam}\\
\affaddr{The Netherlands}\\
\email{ielhelw@cs.vu.nl}
\alignauthor{Rutger Hofman}
\affaddr{Department of Computer Science}\\
\affaddr{Vrije Universiteit Amsterdam}\\
\affaddr{The Netherlands}\\
\email{ielhelw@cs.vu.nl}
\alignauthor{Henri E. Bal}
\affaddr{Department of Computer Science}\\
\affaddr{Vrije Universiteit Amsterdam}\\
\affaddr{The Netherlands}\\
\email{ielhelw@cs.vu.nl}
}

% conference papers do not typically use \thanks and this command
% is locked out in conference mode. If really needed, such as for
% the acknowledgment of grants, issue a \IEEEoverridecommandlockouts
% after \documentclass

% make the title area
\maketitle

\begin{abstract}
The abstract goes here. DO NOT USE SPECIAL CHARACTERS, SYMBOLS, OR MATH IN YOUR TITLE OR ABSTRACT.

\end{abstract}

\category{A.0}{Applications}{Combinatorial and Data Intensive Application; Performance Analysis; Computational Science and Engineering Applications; Algorithms for Accelerators and Heterogeneous Systems}

\section{Introduction}
The tremendous amount of data that we generate through our daily applications such as social networking services, online shopping, and news recommendations, provides us with an opportunity to extract hidden but useful, even invaluable information. Realizing this opportunity, however, requires a significant amount of effort because traditional machine learning algorithms often become extremely inefficient with large amounts of data.

There have been two main approaches to resolving this issue; machine learning researchers have developed new scalable algorithms~\cite{bottou2010large, boyd2011distributed}, while systems and networking researchers have worked on developing new generic infrastructure systems which can be leveraged to construct machine learning solutions more efficiently~\cite{dean2008mapreduce, chang2008bigtable}.
However, the best possible performance is often achieved by carefully integrating both of these approaches in a single solution.

One such big data problem is analyzing large graphs such as social networks where it is not unusual to see a network consisting of billions of edges and tens of million of vertices~\cite{yang2015defining}. In particular, we are interested in the overlapping community detection problem~\cite{xie2013overlapping}, where the goal is to learn the probability distribution of each vertex to participate in each community, given a set of vertices, the links between them (which are usually very sparse), and the number of latent communities. A community can be seen as a densely connected group of vertices that are only sparsely connected to the rest of the network. This problem is significantly more complex than the related domain of detecting disjoint communities.

The problem of detecting overlapping communities is modeled by the mixed membership stochastic block model (MMSB)~\cite{airoldi2009mixed} and in this paper we are particularly interested in a variant of MMSB, called assortative MMSB (a-MMSB\footnote{Although we work on a-MMSB for simplicity, it is also straightforward to apply the proposed method to the general MMSB model.}) \cite{gopalan2012scalable}.
The MMSB model is a probabilistic graphical model~\cite{koller2009probabilistic} that represents a convenient paradigm for modeling complex relationships between a potentially large number of random variables. Bayesian graphical models, where we define priors and infer posteriors over parameters, also allow us to quantify model uncertainty and facilitate model selection and averaging. However, an increasingly urgent question is whether these models and their inference procedures will be up to the challenge of handling very large graphs.

There have been two main recent advances in this direction of scalable Bayesian inference: methods based on stochastic variational Bayes (SVB)~\cite{gopalan2012scalable,hoffman2013stochastic,gopalan2013efficient} and stochastic gradient Markov chain Monte Carlo (SG-MCMC)~\cite{welling2011bayesian,patterson2013stochastic,ahn2014distributed,ahn2012bayesian}. Both methods have the important property that they only require a small subset of the data for every iteration. In other words, they can be applied to (infinite) streaming data.

In this paper, we focus on the SG-MCMC method applied to the a-MMSB model. Recent
work of the authors of this paper~\cite{LiAW15} showed that this methodology is faster and more accurate than the
SVB method. That work further proposes a heuristic to scale up the number of
communities at the cost of less precision; the work in this paper considers the
algorithm without that heuristic.

From a computational point of view, our SG-MCMC a-MMSB algorithm differs
from widespread machine learning algorithms like deep learning in several
ways. First, it is \emph{highly data-intensive} which makes parallel acceleration
particularly challenging. Second, owing to the algorithm's stochastic
nature, the majority of its memory access patterns and data dependencies
are non-deterministic. As a result, straightforward optimization attempts
of the memory access patterns either fail or lead to non-intuitive results. Third,
the size of its intermediate data structures makes it difficult to scale to very large
community graphs.

Starting from an existing implementation in Python/Numpy, we create an efficient
C++ version to serve as the baseline for the parallel versions.
This already improves performance by a factor of 1000 to 1500 compared to
Python.
% This transformation is not considered one of our key contributions.
Then, our key contributions to the algorithm are:

\begin{itemize}
\item \textbf{scaling up to run on GPUs.}
The application is not a natural fit for GPUs, because it is data-bound and
irregular in its data accesses. Nevertheless, we developed aggressive
memory allocation optimizations to solve the data bandwidth issues for GPUs.
This implementation achieves up to 86$\times$ speedup compared to the baseline.
%
We managed to reduce the algorithm's memory requirements by roughly 75\%, which
significantly reduces the data intensity and allows for tackling larger
problems; % while maintaining all state in memory.

\item \textbf{scaling out to overcome the memory bounds posed by GPUs} or,
generally, single-machine systems, to solve the largest publicly available
community graphs. % This necessitated overcoming several challenges.
% The algorithm's state grows rapidly with larger graphs and number of latent
% communities.
Since the full state for the larger graphs does not fit in a
single machine's memory, it is partitioned across a cluster
of machines. Hence, the cluster nodes must read
remote memory hosted by their peers. Since the algorithm is data-bound, we
substantially optimized remote memory accesses by implementing
a custom RDMA store.
On top of that, we pipelined the algorithm stages by fetching
data in advance over the network;
\begin{comment} Finally, the algorithm's computation
is effectively distributed across the cluster nodes and parallelized further
within each node by exploiting their multi-core CPUs.
\end{comment}

\item \textbf{the use of pipelining} \emph{in the distributed implementation}
justifies straightforward intra-node thread parallelism, and it is pointless
to combine the GPU implementations
into it. The cause is that computing for an iteration takes less
time than fetching the data from remote memory, and the computation is
fully overlapped with fulfilling the data requests for the next iteration.
Hence, it is useless to improve the computation time.

\end{itemize}
% To this end, we propose a design of a parallel and distributed system specifically tailored to solve the a-MMSB problem. In particular, we use a mixture of OpenMP, MPI and RDMA in order to efficiently scale and accelerate the algorithm's computation.

\begin{comment}
Further, by cataloguing and accounting for the various load and store
operations, we identified the highest priority locations of data reuse. In
order to circumvent the unclear optimization landscape, we developed an
effective kernel code generation mechanism that explores the benefits of
exploiting all permutations of the available optimization opportunities. These
optimizations include caching in shared memory, caching in the register file,
loop unrolling and explicit vectorization.
\end{comment}

A bird's eye-view of the work in this paper, without any technical detail, has been
published at CCGrid-2016~\cite{10.1109/CCGrid.2016.98}.
The results of our distributed MCMC-aMMSB implementation
have been previously published at the 2016 ParLearning
workshop~\cite{DBLP:conf/ipps/El-HelwHLAWB16}, where it earned a best-paper
award.

The remainder of this paper is organized as follows.
Section~\ref{sec-background} provides an overview of the
algorithm and its theoretical foundation, with related work in
Section~\ref{sec-related}. Section~\ref{sec-design-sequential} describes how we
arrived at an efficient sequential version that serves as the baseline for the
parallel implementations. Section~\ref{sec-gpu} delves
into the parallelization optimizations for GPUs.
Section~\ref{sec-distr} describes the design, implementation and evaluation of
the scalable distributed solution. Finally, Section~\ref{sec-conclusion}
provides concluding remarks.

% -------\\
% Community detection is the central problem in network analysis, with the goal of identifying the groups of related nodes that are densely connected within this group but sparsely connected to the rest of the network. Different from classical community detection problem where we assume each node belongs to one single community, our paper considers overlapping communities where each of the nodes might belong to multiple communities. In particular, we consider the model called a-MMSB(Assortive Mixed Membership Stochastic Blockmodel) in this paper, which was first introduced in~\cite{gopalan2012scalable}.\\


% a-MMSB, as a probabilistic graphical model, represents a convenient paradigm for modeling complex relationships between a potentially large number of random variables. It also uses priors and posteriors to quantify model uncertainty and facilitate model selection and averaging. While a-MMSB provides rich representation power, like other Bayesian models, the inference procedures of handling big data which is common in real world is still a very challenging problem. \\



% Consider the large networks such as a social network, it easily runs into billions of edges and tens of million of nodes. In addition to that, the number of communities might exceed few millions. 
% There were two types of scalable algorithm for a-MMSB have been introduced recently \cite{gopalan2012scalable}, stochastic variational Bayesian inference (SVB) and stochastic gradient Markov chain Monte Carlo(SG-MCMC), respectively. Both methods have the important property that each iteration only relies on a small subset of the data. Although both methods can work for some large networks with many hundreds of nodes, the performance is far less satisfactory when it applies to the so-called "big data" such as facebook network with millions of nodes and billions of edges. Given the facts that \cite{LiAW15}, SG-MCMC runs faster and converges to the better local minima, in this paper, we mainly consider the problem of scaling up SG-MCMC to the big data set.  \textit{As far as we know, this is the first paper that studies community detection on the frencter dataset with few billion's of edges.}





\section{Related Work}

\subsection{Related work on Accelerators for Bayesian methods}

In this section, we discuss related machine learning projects that use
accelerators.
Immediately when GPUs evolved into generally programmable accelerators, machine
learners recognized the potential of GPUs for their science.
Much research
has been devoted to deep-learning implementations on GPUs. However, Mixed
Membership deduction (MMSB) is usually considered untractable for deep-learning
algorithms. In contrast, our algorithm belongs to the family of Bayesian
(approximative, stochastic) methods, where efficient algorithms for large-scale
data have recently been developed. A broad survey of Bayesian methods is given
in~\cite{DBLP:journals/corr/Zhu0H14}. Broadly speaking, Bayesian approximative
algorithms are subdivided into variational and Monte Carlo methods; Markov
Chain Monte Carlo (MCMC) is a subgroup of the latter. GPUs have been put to
use in MCMC algorithms, e.g.\ \cite{journals/bioinformatics/MedlarGSBK13}
uses GPUs to analyse parental linkage patterns in a biology context and
\cite{DBLP:journals/csda/WhiteP14} does the same to model terrorist activity.
Variational inference on MMSB is described
in~\cite{DBLP:conf/nips/GopalanMGFB12}, but they don't use GPUs. Latent
Dirichlet Allocation, another variety of Bayesian Approximation, is used on
GPUs by~\cite{DBLP:conf/nips/YanXQ09}.

Still, for high-dimension problems, the standard MCMC algorithms fall short in
performance. Recently, varieties of MCMC algorithms have been developed where
gradient information is used to speed up convergence. Langevin and Hamiltonian
dynamics are representatives of these
varieties~\cite{Girolami_riemannmanifold}.  Our algorithm uses Riemann Manifold
Langevin dynamics. In \cite{beam2014fast}, GPUs are used to perform Hamiltonian
descent, using Python interfaces to access the standard cuBLAS
library~\cite{cuBLAS}. They limit their optimizations for the GPU to
saving on data transfers between host and device memory. They do not apply
their framework to MMSB.

A different MMSB algorithm with stochastic gradient descent on the GPU is
the Online Tensor approach~\cite{DBLP:journals/corr/HuangNHVA13}. They deduce
overlapping communities for many of the same datasets as this paper, and they
report fast convergence. Their implementation uses the cuBLAS
library, and, unlike our work, there is no attempt to hand-optimize the GPU
kernels. Additionally, the algorithmic approach to solve the problem is
fundamentally different.
Since they target GPUs only, the datasets they
can handle are limited by the device memory of the GPU. Our implementation converges
quickly on the GPU but can also function with reduced speedup on a multicore
CPU. In the latter case, the dataset
size is only limited by the memory of the host machine.

\begin{comment}
Since our SG-MCMC algorithm is completely different from deep-learning
algorithms, it makes no real sense to compare to the learning frameworks
Theano, Caffe, cuDNN, or NVidia Digits, even though they target GPUs.

SG-MCMC work NOT on GPUs:

GPU work:
MCMC Hamiltonian:
	1. Andrew L. Beam, Sujit K. Ghosh, Jon Doyle
		Fast Hamiltonian Monte Carlo Using GPU Computing
		http://arxiv.org/pdf/1402.4089.pdf
MMSB tensor:

MCMC (not SG):
    1. Alan Medlar, Dorota Głowacka, Horia Stanescu, Kevin Bryson, Robert Kleta
       SwiftLink: Parallel MCMC linkage analysis utilising multicore CPU and GPU
       http://bioinformatics.oxfordjournals.org/content/early/2012/12/13/bioinformatics.bts704.full.pdf
    2. Marc Suchard, Chris Holmes, Mike West
       Some of the What?, Why?, How?, Who?  and Where?  of Graphics Processing Unit Computing for Bayesian Analysis
       https://stat.duke.edu/gpustatsci/GPU-ISBABull2010.pdf
\end{comment}

\begin{comment}
In this paper, we describe our custom RDMA D-KV (Distributed Key-Value)
store. Current RMDA D-KV store implementations are RamCloud~\cite{RamCloud},
Pilaf~\cite{Pilaf}, Herd~\cite{Herd} and FaRM~\cite{FaRM}. All these systems
use RDMA to implement a D-KV store. However, all of them are far more powerful
than our custom implementation -- and this power comes at a cost that we
can avoid. They implement a generic D-KV store that controls concurrency,
supports dynamic inserts and deletes, supports variable-sized values
(whose size may change at an update), and keys of arbitrary type. Because
of the nature of our distributed algorithm, we have to deal with none of
these issues. For us, values are fixed-size, allocated only at the initial
population, and remain alive forever. We have no concurrency between writes
and reads or other writes. Our keys are a contiguous range of integers. All
these properties together allow an extremely low-overhead implementation
that does not involve the remote host in any transaction.
\end{comment}

There are a few projects that did a distributed implementation
for community detection on very large network graphs. In contrast to our work,
they all detect non-overlapping communities: \cite{Bu2013246} investigate a
large number of networks; \cite{2015arXiv150302115L} investigate hierarchical
stochastic blockmodels; \cite{Prat-Perez:2014:HQS:2566486.2568010} use
multi-core machines.

\begin{comment}
\subsection{Related work on RDMA Key-Value stores}
In this paper, we describe our custom RDMA D-KV (Distributed Key-Value)
store. Current RMDA D-KV store implementations are RamCloud~\cite{RamCloud},
Pilaf~\cite{Pilaf}, Herd~\cite{Herd} and FaRM~\cite{FaRM}. All these systems
use RDMA to implement a D-KV store. However, all of them are far more powerful
than our custom implementation -- and this power comes at a cost that we
can avoid. They implement a generic D-KV store that controls concurrency,
supports dynamic inserts and deletes, supports variable-sized values
(whose size may change at an update), and keys of arbitrary type. Because
of the nature of our distributed algorithm, we have to deal with none of
these issues. For us, values are fixed-size, allocated only at the initial
population, and remain alive forever. We have no concurrency between writes
and reads or other writes. Our keys are a contiguous range of integers. All
these properties together allow an extremely low-overhead implementation
that does not involve the remote host in any transaction.
\end{comment}

\input{03-problem-statement}
\input{04-design}
\input{05-evaluation}
% \section{Conclusions}
\label{sec-conclusion}

Modern machine learning algorithms are empowering their
users to solve or approximate solutions to previously intractable problems.
However, these advancements come at the cost of significant computational
complexity and long running learning processes which hinder our ability to reap
such algorithms' benefits when applied to large problems. There is a clear need
to assess such algorithms and identify optimization opportunities in order to
scale their performance.

The SG-MCMC algorithm discussed in this paper posed additional computational
challenges due to its unique stochastic nature and \emph{high data intensity}. Unlike
common machine learning algorithms, its data dependencies and memory access
patterns are \emph{nondeterministic}. In this paper, we presented our methodology of
improving the algorithm's performance by restructuring it to
cater for concurrent and distributed parallelism.

We showed that the algorithm's state can be reduced by 75\% after a thorough
analysis of the computational patterns and data structures, thus significantly
reducing its data intensity.

In our implementation for many-core processors, we navigated the complex optimization
landscape by dynamically generating kernel code and investigating different
combinations of optimizations. The outcome of these efforts culminated in
significant speedup factors of 21 and 120 using a multi-core CPU and a GPU
respectively. These speedup numbers were achieved in comparison to an already
optimized sequential implementation. The evaluation of the performance
across several GPUs yielded one striking find. No single optimization strategy
works best for all GPUs: different choices for memory allocation strategy,
kernel block size and vector width were required for optimal results for
different GPUs and application instances.
%
Hence, the outcome of this study reinforces the significance of avoiding premature
optimization. In particular, the success
of common GPU optimizations depends on the device and the
problem size it is applied to.

We further created a highly scalable implementation to handle the largest existing
community graphs on a distributed cluster machine.
This design solved
several problems to achieve scalability and high performance.
We overlapped computation
with communication to hide latency.  We use a
mixture of MPI and RDMA primitives to speed up the communication between cluster
nodes.
%
Further, we conducted a thorough empirical evaluation of the system to study its strong
and weak scalability on 65 cluster nodes using large data sets.
% Additionally, we assessed the efficiency of the algorithm's resource utilization.
Finally, a
demonstration of the implementation's utility was provided by processing 6
large real-world data sets.
To the best of
our knowledge, this is the first time that the problem of deducing overlapping
communities has been learned for problems of such a large scale.

% use section* for acknowledgement
\section*{Acknowledgment}

The authors would like to thank...
more thanks here


% trigger a \newpage just before the given reference
% number - used to balance the columns on the last page
% adjust value as needed - may need to be readjusted if
% the document is modified later
%\IEEEtriggeratref{8}
% The "triggered" command can be changed if desired:
%\IEEEtriggercmd{\enlargethispage{-5in}}

% references section

% can use a bibliography generated by BibTeX as a .bbl file
% BibTeX documentation can be easily obtained at:
% http://www.ctan.org/tex-archive/biblio/bibtex/contrib/doc/
% The IEEEtran BibTeX style support page is at:
% http://www.michaelshell.org/tex/ieeetran/bibtex/
%\bibliographystyle{IEEEtran}
% argument is your BibTeX string definitions and bibliography database(s)
%\bibliography{IEEEabrv,../bib/paper}
%
% <OR> manually copy in the resultant .bbl file
% set second argument of \begin to the number of references
% (used to reserve space for the reference number labels box)
\begin{thebibliography}{1}

\bibitem{IEEEhowto:kopka}
H.~Kopka and P.~W. Daly, \emph{A Guide to \LaTeX}, 3rd~ed.\hskip 1em plus
  0.5em minus 0.4em\relax Harlow, England: Addison-Wesley, 1999.

\end{thebibliography}




% that's all folks
\end{document}


