\documentclass[10pt, conference, compsocconf]{IEEEtran}

% *** GRAPHICS RELATED PACKAGES ***
%
\ifCLASSINFOpdf
  % \usepackage[pdftex]{graphicx}
  % declare the path(s) where your graphic files are
  % \graphicspath{{../pdf/}{../jpeg/}}
  % and their extensions so you won't have to specify these with
  % every instance of \includegraphics
  % \DeclareGraphicsExtensions{.pdf,.jpeg,.png}
\else
  % or other class option (dvipsone, dvipdf, if not using dvips). graphicx
  % will default to the driver specified in the system graphics.cfg if no
  % driver is specified.
  % \usepackage[dvips]{graphicx}
  % declare the path(s) where your graphic files are
  % \graphicspath{{../eps/}}
  % and their extensions so you won't have to specify these with
  % every instance of \includegraphics
  % \DeclareGraphicsExtensions{.eps}
\fi
% graphicx was written by David Carlisle and Sebastian Rahtz. It is
% required if you want graphics, photos, etc. graphicx.sty is already
% installed on most LaTeX systems. The latest version and documentation can
% be obtained at: 
% http://www.ctan.org/tex-archive/macros/latex/required/graphics/
% Another good source of documentation is "Using Imported Graphics in
% LaTeX2e" by Keith Reckdahl which can be found as epslatex.ps or
% epslatex.pdf at: http://www.ctan.org/tex-archive/info/
%
% latex, and pdflatex in dvi mode, support graphics in encapsulated
% postscript (.eps) format. pdflatex in pdf mode supports graphics
% in .pdf, .jpeg, .png and .mps (metapost) formats. Users should ensure
% that all non-photo figures use a vector format (.eps, .pdf, .mps) and
% not a bitmapped formats (.jpeg, .png). IEEE frowns on bitmapped formats
% which can result in "jaggedy"/blurry rendering of lines and letters as
% well as large increases in file sizes.
%
% You can find documentation about the pdfTeX application at:
% http://www.tug.org/applications/pdftex





% *** MATH PACKAGES ***
%\usepackage[cmex10]{amsmath}

% *** SPECIALIZED LIST PACKAGES ***
%
% \usepackage{algorithmic}
% algorithmic.sty was written by Peter Williams and Rogerio Brito.
% This package provides an algorithmic environment fo describing algorithms.
% You can use the algorithmic environment in-text or within a figure
% environment to provide for a floating algorithm. Do NOT use the algorithm
% floating environment provided by algorithm.sty (by the same authors) or
% algorithm2e.sty (by Christophe Fiorio) as IEEE does not use dedicated
% algorithm float types and packages that provide these will not provide
% correct IEEE style captions. The latest version and documentation of
% algorithmic.sty can be obtained at:
% http://www.ctan.org/tex-archive/macros/latex/contrib/algorithms/
% There is also a support site at:
% http://algorithms.berlios.de/index.html
% Also of interest may be the (relatively newer and more customizable)
% algorithmicx.sty package by Szasz Janos:
% \usepackage[noendl]{algorithmicx}
% \usepackage[noend]{algpseudocode}
% http://www.ctan.org/tex-archive/macros/latex/contrib/algorithmicx/
% \renewcommand{\algorithmiccomment}[1]{\hskip 3em// \textit{#1}}

\usepackage{listings}
\lstset{
	language=C,
	morekeywords={iterate, until, for, every},
	basicstyle=\small\rmfamily,
	keywordstyle=\bfseries,
	numbers=left,
	columns=fullflexible,
	showstringspaces=false,
	xleftmargin=1.8em,
	frame=lines,
}




% *** ALIGNMENT PACKAGES ***
%
\usepackage{array}
% Frank Mittelbach's and David Carlisle's array.sty patches and improves
% the standard LaTeX2e array and tabular environments to provide better
% appearance and additional user controls. As the default LaTeX2e table
% generation code is lacking to the point of almost being broken with
% respect to the quality of the end results, all users are strongly
% advised to use an enhanced (at the very least that provided by array.sty)
% set of table tools. array.sty is already installed on most systems. The
% latest version and documentation can be obtained at:
% http://www.ctan.org/tex-archive/macros/latex/required/tools/


%\usepackage{mdwmath}
%\usepackage{mdwtab}
% Also highly recommended is Mark Wooding's extremely powerful MDW tools,
% especially mdwmath.sty and mdwtab.sty which are used to format equations
% and tables, respectively. The MDWtools set is already installed on most
% LaTeX systems. The lastest version and documentation is available at:
% http://www.ctan.org/tex-archive/macros/latex/contrib/mdwtools/


% IEEEtran contains the IEEEeqnarray family of commands that can be used to
% generate multiline equations as well as matrices, tables, etc., of high
% quality.


%\usepackage{eqparbox}
% Also of notable interest is Scott Pakin's eqparbox package for creating
% (automatically sized) equal width boxes - aka "natural width parboxes".
% Available at:
% http://www.ctan.org/tex-archive/macros/latex/contrib/eqparbox/





% *** SUBFIGURE PACKAGES ***
%\usepackage[tight,footnotesize]{subfigure}
% subfigure.sty was written by Steven Douglas Cochran. This package makes it
% easy to put subfigures in your figures. e.g., "Figure 1a and 1b". For IEEE
% work, it is a good idea to load it with the tight package option to reduce
% the amount of white space around the subfigures. subfigure.sty is already
% installed on most LaTeX systems. The latest version and documentation can
% be obtained at:
% http://www.ctan.org/tex-archive/obsolete/macros/latex/contrib/subfigure/
% subfigure.sty has been superceeded by subfig.sty.



%\usepackage[caption=false]{caption}
%\usepackage[font=footnotesize]{subfig}
% subfig.sty, also written by Steven Douglas Cochran, is the modern
% replacement for subfigure.sty. However, subfig.sty requires and
% automatically loads Axel Sommerfeldt's caption.sty which will override
% IEEEtran.cls handling of captions and this will result in nonIEEE style
% figure/table captions. To prevent this problem, be sure and preload
% caption.sty with its "caption=false" package option. This is will preserve
% IEEEtran.cls handing of captions. Version 1.3 (2005/06/28) and later 
% (recommended due to many improvements over 1.2) of subfig.sty supports
% the caption=false option directly:
%\usepackage[caption=false,font=footnotesize]{subfig}
%
% The latest version and documentation can be obtained at:
% http://www.ctan.org/tex-archive/macros/latex/contrib/subfig/
% The latest version and documentation of caption.sty can be obtained at:
% http://www.ctan.org/tex-archive/macros/latex/contrib/caption/




% *** FLOAT PACKAGES ***
%
\usepackage{fixltx2e}
% fixltx2e, the successor to the earlier fix2col.sty, was written by
% Frank Mittelbach and David Carlisle. This package corrects a few problems
% in the LaTeX2e kernel, the most notable of which is that in current
% LaTeX2e releases, the ordering of single and double column floats is not
% guaranteed to be preserved. Thus, an unpatched LaTeX2e can allow a
% single column figure to be placed prior to an earlier double column
% figure. The latest version and documentation can be found at:
% http://www.ctan.org/tex-archive/macros/latex/base/



\usepackage{stfloats}
% stfloats.sty was written by Sigitas Tolusis. This package gives LaTeX2e
% the ability to do double column floats at the bottom of the page as well
% as the top. (e.g., "\begin{figure*}[!b]" is not normally possible in
% LaTeX2e). It also provides a command:
%\fnbelowfloat
% to enable the placement of footnotes below bottom floats (the standard
% LaTeX2e kernel puts them above bottom floats). This is an invasive package
% which rewrites many portions of the LaTeX2e float routines. It may not work
% with other packages that modify the LaTeX2e float routines. The latest
% version and documentation can be obtained at:
% http://www.ctan.org/tex-archive/macros/latex/contrib/sttools/
% Documentation is contained in the stfloats.sty comments as well as in the
% presfull.pdf file. Do not use the stfloats baselinefloat ability as IEEE
% does not allow \baselineskip to stretch. Authors submitting work to the
% IEEE should note that IEEE rarely uses double column equations and
% that authors should try to avoid such use. Do not be tempted to use the
% cuted.sty or midfloat.sty packages (also by Sigitas Tolusis) as IEEE does
% not format its papers in such ways.





\usepackage{url}
\makeatletter
\g@addto@macro{\UrlBreaks}{\UrlOrds}
\makeatother

% correct bad hyphenation here
\hyphenation{op-tical net-works semi-conduc-tor}

% from SuperComputing Glasswing paper
\usepackage{epsfig}
\usepackage{amssymb}
\usepackage{amsmath}
\usepackage{amsfonts}
\usepackage{pslatex}
% \usepackage{comment}
\usepackage{verbatim}
\usepackage{dcolumn}
\newcolumntype{d}[1]{D{.}{.}{#1}}
\usepackage[noadjust]{cite}
\usepackage{fancyhdr}
\setcounter{secnumdepth}{4}

\newcommand\Note[1]{\textbf{Note: #1}}

\begin{document}
%
% paper title
% can use linebreaks \\ within to get better formatting as desired
\title{Detecting Overlapping Communities within Graphs using an Accelerated Markov
Chain Monte Carlo Algorithm}
\title{Performance Tunning a Stochastic Gradient Markov Chain Monte Carlo
Overlapping Community Detection Algorithm}

\title{Optimizing a Markov Chain Monte Carlo Algorithm to Detect Overlapping
Communities}

\title{Accelerating Overlapping Community Detection: Performance Tuning
 a Stochastic Gradient Markov Chain Monte Carlo Algorithm}


% author names and affiliations
% use a multiple column layout for up to two different
% affiliations
\author{\IEEEauthorblockN{Ismail El-Helw}
\IEEEauthorblockA{Department of Computer Science\\
Vrije Universiteit Amsterdam\\
The Netherlands\\
ielhelw@cs.vu.nl}
\and
\IEEEauthorblockN{Rutger Hofman}
\IEEEauthorblockA{Department of Computer Science\\
Vrije Universiteit Amsterdam\\
The Netherlands\\
rutger@cs.vu.nl}
\and
\IEEEauthorblockN{Henri E. Bal}
\IEEEauthorblockA{Department of Computer Science\\
Vrije Universiteit Amsterdam\\
The Netherlands\\
bal@cs.vu.nl}
}

% conference papers do not typically use \thanks and this command
% is locked out in conference mode. If really needed, such as for
% the acknowledgment of grants, issue a \IEEEoverridecommandlockouts
% after \documentclass

% make the title area
\maketitle

\begin{abstract}
Accelerating sequential algorithms in order to achieve high performance is
often a nontrivial task. However, there are certain properties that can
exacerbate this process and make it particularly daunting. For example,
building an efficient parallel solution for a data-intensive algorithm requires
a deep analysis of the memory access patterns and data reuse potential. In this
context, the optimization landscape can be extremely complex owing to the large
number of trade-off decisions.

In this paper, we discuss our experience accelerating an existing
data-intensive machine learning algorithm that detects overlapping communities
in graphs. We present our methodology and findings in parallelizing the
computation using either a multi-core CPU or one of several GPUs. A thorough
analysis of the algorithm's computation and data dependencies resulted in a
75\% reduction in its memory footprint and data intensity. Additionally, we
employed a customizable code generation strategy to test and identify the
optimal combination of optimizations for each compute device. We present an
empirical evaluation of these contributions and show that the parallelization
achieves speedups up to 86x compared to an already optimized sequential
version.
\end{abstract}

\begin{IEEEkeywords}
Algorithms for Accelerators and Heterogeneous Systems;
Performance Analysis;
Combinatorial and Data Intensive Application;
Statistical Learning.

\end{IEEEkeywords}


% For peer review papers, you can put extra information on the cover
% page as needed:
% \ifCLASSOPTIONpeerreview
% \begin{center} \bfseries EDICS Category: 3-BBND \end{center}
% \fi
%
% For peerreview papers, this IEEEtran command inserts a page break and
% creates the second title. It will be ignored for other modes.
\IEEEpeerreviewmaketitle

\section{Introduction}
\cite{DBLP:journals/corr/LiAW15}

\input{03-problem-statement}
\input{04-design}
\input{05-evaluation}
\section{Background}
% In this section, we briefly review the Bayesian model we are considering in this paper.

% In this section, we briefly review the Bayesian model we are considering in this paper.
\subsection{Assortative Mixed-membership Stochastic Blockmodels (a-MMSB)}
The assortative mixed membership stochastic blockmodel (a-MMSB)~\cite{gopalan2012scalable} is a special case of MMSB~\cite{airoldi2009mixed} that models the group structure in a network of $N$ vertices. Consider a set  $\cV^*$ containing all the vertices in the graph, and a set $\cE^*$ containing all the linked edges between pairs of vertices. Each vertex $a$ in the vertex set $\cV^*$ has a $K$-dimensional probability distribution $\pi_a$ of participating in the $K$ members of the community set $\cK$.  For every possible peer $b$ in the network, each vertex $a$ randomly draws a community $z_{ab}$. If a pair of vertices ($a,b$) in the edge set $\cE^*$ are in the same community, i.e., $z_{ab}=z_{ba} = k$, then they have a significant probability $\bt_k$ to connect, i.e., $y_{ab}=1$. Otherwise this probability is small. Each community has its connection strength $\beta_{k} \in (0,1)$ which reflects how likely its members are linked to each other. The whole generative process of a-MMSB is then described by:
\benum
\item For each community $k$, draw community strength $\bt_k \sim \mbox{Beta}(\eta)$
\item For each vertex $a$, draw community memberships $\pi_a \sim \mbox{Dirichlet}(\al)$
\item For each pair of vertices $a$ and $b$,
\benum
	\item Draw interaction indicator $z_{ab} \sim \pi_a$
    \item Draw interaction indicator $z_{ba} \sim \pi_b$
    \item Draw link $y_{ab} \sim \mbox{Bernoulli}(r)$, where $r = \bt_k$ if $z_{ab}=z_{ba}=k$, and $r=\dt$ otherwise.
\eenum
\eenum
$\eta$ and $\alpha$ are parameters to the Beta and Dirichlet distribution functions. $\dt$ (usually small) is a parameter for the algorithm.

\newcommand{\Vertices}{\ensuremath{\cV^*{}}\xspace}
\newcommand{\Edges}{\ensuremath{\cE^*{}}\xspace}
\newcommand{\Heldout}{\ensuremath{\cE_h{}}\xspace}
\newcommand{\SGLDMinibatch}{\ensuremath{\cD_n{}}\xspace}
\newcommand{\Minibatch}{\ensuremath{\cE_n{}}\xspace}
\newcommand{\Minibatcht}{\ensuremath{\cE_{n_t}{}}\xspace}
\newcommand{\Neighbors}{\ensuremath{\cV_n{}}\xspace}

\subsection{Stochastic Gradient Markov Chain Monte Carlo}
The algorithm we are working on in this paper is based on stochastic gradient Langevin dynamics (SGLD)~\cite{welling2011bayesian}. SGLD applies the following update rule to obtain samples from a posterior distribution $p(\ta|\cX) \propto p(\cX|\ta) p(\ta)$ of $N$ i.i.d. data points $\cX=\{x_i\}_{i=1}^{N}$:
\bea
\ta^* \law \ta + \f{\epsilon_t}{2}\left(\nabla_{\ta}\log p(\ta_t)+N\bar{g}(\ta;\SGLDMinibatch)\right) + \xi_t, \label{eqn:sgld_update}
\eea
where $\xi_t \sim \cN(0,\epsilon_t )$ with $\ep_t$ the step size, $\SGLDMinibatch$ a mini-batch of size $n$ sampled from $\cX$, and $\bar{g}(\ta;\SGLDMinibatch)$ the mean stochastic gradient, i.e., $\frac{1}{|\SGLDMinibatch|}\sum_{x\in \SGLDMinibatch}^{}\grad_{\ta} \log p(x|\ta)$. As the step size goes to zero by a schedule satisfying $\sm{t}{\infty}\ep_t = \infty$ and $\sm{t}{\infty}\ep_t^2 < \infty$, SGLD samples from the true posterior distribution. One benefit of using SGLD is that we do not need the Metropolis-Hastings (MH) accept-reject tests since the rejection probability goes to zero as the step size collapses to zero. Although the finite step size, which we use in practice rather than converging to zero, still results in some bias, we can reduce the overall error by drawing many more samples per unit time. 

SGLD originated from the Langevin Monte Carlo (LMC)~\cite{girolami2011riemann}, where unlike SGLD the gradient is computed exactly by using all data points. Then, Metropolis-Hastings accept-reject tests are applied. Comparing to LMC, SGLD only requires to process a mini-batch $\SGLDMinibatch$ at each iteration and ignores the MH test, and thus the computational complexity substantially reduces from $\cO(N)$ to $\cO(n)$.   

Stochastic gradient Riemannian Langevin dynamics (SGRLD)~\cite{patterson2013stochastic} is a subclass of SGLD which is developed to efficiently sample from the probability simplex. By applying Riemannian geometry~\cite{girolami2011riemann} and using the mini-batch-based estimator in Eqn. \ref{eqn:sgld_update}, it achieved state-of-the-art performance for latent Dirichlet allocation (LDA)~\cite{blei2003latent}. In particular, for a $K$-dimensional probability simplex $\pi$, it uses the \textit{expanded-mean} re-parameterization trick, where the probability of a category $k$ is given by $\pi_k = \ta_k / \sum_{j=1}^K \ta_j$ with $\ta_k \sim \text{Gamma}(\al,1)$ and $\al$ a hyperparameter of the Dirichlet distribution $p(\pi | \al )$. Then, the update rule of SGRLD is:
\bea
\ta_{k}^* \law \left| \ta_{k} + \f{\ep_t}{2} \left( \al - \ta_{k} + \f{N}{|\SGLDMinibatch|} \sum_{d \in \SGLDMinibatch} g_d(\ta_{k}) \right) + (\ta_{k})^\ha \xi_{t} \right|, \label{eqn:sgrld_update}
\eea
here $g_d(\ta_{k})$ is the gradient of the log posterior w.r.t. $\ta_{k}$ on a data point $d\in \SGLDMinibatch$.


\subsection{Scalable MCMC for a-MMSB}
This section describes the SG-MCMC algorithm for a-MMSB. Please refer \cite{LiAW15} for more details. The algorithm iterates updating local parameter $\pi$ and global parameter $\beta$. Since both parameters lie on the probability simplex, SGRLD is applied to make the sampling process more efficient. Here, the parameters $\phi$ and $\ta$ are used to re-parameterize $\pi$ and $\beta$ respectively. After updating $\phi$ and $\ta$, we can obtain $\pi$ and $\beta$ by normalizing $\phi$ and $\ta$, respectively. In the following, we briefly sketch the iterative update steps.

\paragraph{\textbf{Sampling global parameters}} The update rule for global parameter $\theta$ is
\bea
\ta_{ki}^* \law \left| \ta_{ki} + \f{\ep}{2} \left( \eta - \ta_{ki} + h(\Minibatcht)\sum_{(a,b) \in \Minibatcht}g_{ab}(\ta_{ki})\right) \right.\nn \\ \left.+ (\ta_{ki})^{\ha}\xi_{ki} \right|, \label{eqn:global_update}
\eea
where the gradient in $\theta$
\bea
g_{ab}(\ta_{ki})
&=& \f{f_{ab}^{(y)}(k,k)}{Z_{ab}^{(y)}} \left(\f{|1-i-y|}{\ta_{ki}} - \f{1}{\sum_j\ta_{kj}} \right),
\eea
here $\Minibatcht$ is a mini-batch of $n_t$ vertex pairs sampled from $\Edges$ or $E$; $h(\Minibatcht)$ is a weight factor to scale the effect of a mini-batch towards the full network; and 
\bea
f_{ab}^{(y)}(k,l)=
\begin{cases}
\bt_k^y(1-\bt_k)^{(1-y)}\pi_{ak}\pi_{bk}, & \mbox{if } k=l\\
\dt^{y}(1-\dt)^{(1-y)} \pi_{ak} \pi_{bl}, & \mbox{if } k \neq l \nn.
\end{cases}
\label{eqn:case}
\eea
$Z_{ab}^{(y)}$ is the normalization constant which we can compute in $\cO(K)$ time \cite{LiAW15}.

\paragraph{\textbf{Sampling local parameters}}
The update rule for the local parameters $\phi$ is
\bea
\phi_{ak}^* \law \left| \phi_{ak} + \f{\ep}{2} \left( \al - \phi_{ak} + \f{N}{|\Neighbors|} \sum_{b \in \Neighbors} g_{ab}(\phi_{ak})\right) \right. \nn \\ \left. + (\phi_{ak})^{\ha} \xi_{ak}\right|,
\label{eqn:local_update}
\eea
where the gradient in $\phi$
\bea
g_{ab}(\phi_{ak}) = \f{f_{ab}^{(y)}(k)}{Z_{ab}^{(y)}\phi_{ak}} - \f{1}{\sum_j\phi_{aj}}.
\eea
Here, $\Neighbors$ is the neighbor set for a mini-batch node, another random mini-batch of $n$ nodes sampled from $\cV^*$. Note that $|\Neighbors| \ll |\cV^*|=N$.

As is common in machine learning problems, the edges of the graph are divided
into two sets, the training set and a held-out set \Heldout.
As the performance metric we use perplexity, which is defined as the exponential
of the negative average log-likelihood of the held-out set \Heldout. Given a collection of $T$
samples of the model parameters $\{\bt_t\}$ and $\{\pi_t\}$, the averaged
perplexity on the held-out test set $\Heldout$ is 
\begin{align}
&\text{perp}_{\text{avg}}(\Heldout|\{\beta_t\},\{\pi_t\})\nonumber \\=&
\exp\left(-\frac{\sum_{(a,b)\in \Heldout}^{}\log
\{(1/T)\sm{t}{T}p(y_{ab}|\beta_t,\pi_t)\}}{|\cE_{h}|}\right)
\end{align}
The perplexity is not evaluated at every iteration, but at regular intervals.

The pseudo-code of the sequential algorithm is presented in Algorithm~\ref{algorithm}. 

\begin{algorithm}[t]
\caption{Sequential version of SG-MCMC for a-MMSB}\label{alg}
\begin{algorithmic}[1]
\STATE Initialize $\pi , \bt, \phi, \ta$
\WHILE {sampling}
\STATE Sample a mini-batch of vertex pairs, \Minibatch, from $E$
\FOR {each vertex in \Minibatch}
	\STATE Sample a mini-batch of vertices, $\Neighbors$, from $\Vertices$
    \STATE Update $\phi_{a}$ using Eqn~\ref{eqn:local_update}
  %  $\phi_{a}^* \law \left| \phi_{a} + \f{\ep_{\phi}}{2} \left( \al - \phi_{a} + \f{N}{|\Neighbors|} \sum_{b \in \Neighbors} G_{\phi_{a}}\right) + (\phi_{a})^{\ha} \xi_{a}\right|$
    \STATE Obtain $\pi_a$ from $\phi_a^*$ \ENDFOR
\FOR {$k = 1,\dots,K$}
    \STATE Update $\ta_{k}$ using Eqn~\ref{eqn:global_update}
   % $\ta_{k}^* \law \left| \ta_{k} + \f{\ep_\ta}{2} \left( \eta - \ta_{k} + h(\cE_t) \sum_{(a,b) \in \cE_t} G_{\ta_{k}}(a,b) \right) + (\ta_{k})^{\ha}\xi_{k} \right|$
    \STATE Obtain $\bt_k$ from $\ta_k^*$
\ENDFOR
\ENDWHILE
\end{algorithmic}
\label{algorithm}
\end{algorithm}


\addtolength{\tabcolsep}{-3pt}
\renewcommand{\arraystretch}{1.5}
\begin{table}[b] % [htbp]
\center
\begin{tabular}{c C{0.075\textwidth} c L{0.26\textwidth}}
\textbf{symbol}
	 & \textbf{type} & \textbf{size} & \textbf{description} \\
         % & \multicolumn{1}{c}{\textbf{type}}
                           % & \multicolumn{1}{c}{\textbf{description}} \\
\hline
$\cK$      &              & $K$        & set of communities \\
\Vertices  & \{vertex\}   & $N$          & vertices in the graph \\
\Edges     & \{edge\}     &              & linked edges in the graph \\
$E$        & \{edge\}     &              & $\Vertices\times\Vertices$: linked and nonlinked edges \\
\Heldout   & \{edge\}     &              & held-out subset of the graph \\
\Minibatch & \{edge\}     &              & sampled mini-batch of edges in~$E$ \\
$M$        &              &              & number of vertices in \Minibatch \\
\Neighbors & \{vertex\}   &              & sampled neighbor set for a vertex in~\Minibatch \\
$\theta$   & float~vector 2-D & $K\times{}2$ & global latent variables for community strength \\
$\beta$    & float vector & $K$          & $\beta[k] = \theta[k][0] / \sum_j\theta[k][j]$ is the community strength\\
$\phi$     & float~vector 2-D & $N\times{}K$ & local latent variables; $\phi[i][k]$ reflects
                            probability that vertex $i$ is in community~$k$ \\
$\pi$      & float~vector 2-D & $N\times{}K$ &
					$\pi[i][k]$~is~probability that $i$ is in $k$
					$\pi[i][k] = \phi[i][k] / \sum_j\phi[i][j]$ \\
\hline
\end{tabular}
\caption{Definition of most important symbols}
\label{tbl-symbols}
\end{table}
\addtolength{\tabcolsep}{3pt}
\renewcommand{\arraystretch}{1.0}

\begin{comment}
Assortative mixed-membership stochastic blockmodel (a-MMSB) [1] is a special case of MMSB [8] that models the group-structure in a network of $N$ nodes. In particular, each node $a$ in the node set $\cV^*$ has a $K$-dimension probability distribution $\pi_a$ of participating in the $K$ members of the community set $\cK$. For every possible peer $b$ in the network, each node $a$ randomly draws a community $z_{ab}$. If a pair of nodes ($a,b$) in the edge set $\cE^*$ are in the same community: $z_{ab}=z_{ba} = k$, then they have a significant probability $\bt_k$ to connect, i.e., $y_{ab}=1$. Otherwise this probability is small. Each community has its connection strength $\beta_{k} \in (0,1)$ which explains how likely its members are linked to each other.  

The generative process of a-MMSB is then given by,

\benum
\item For each community $k$, draw community strength $\bt_k \sim \mbox{Beta}(\eta)$
\item For each node $a$, draw community memberships $\pi_a \sim \mbox{Dirichlet}(\al)$
\item For each pair of nodes $a$ and $b$,
\benum
	\item Draw interaction indicator $z_{ab} \sim \pi_a$
    \item Draw interaction indicator $z_{ba} \sim \pi_b$
    \item Draw link $y_{ab} \sim \mbox{Bernoulli}(r)$, where $r = \bt_k$ if $z_{ab}=z_{ba}=k$, and $r=\dt$ otherwise.
\eenum
\eenum

Unlike the a-MMSB, the original MMSB maintains pair-wise community strength $\bt_{k,k'}$ for all pairs of the communities. Note that it is trivial to extend the results that we obtain in this paper to the general MMSB model. The joint probability of the above process can be written as:
\begin{align}
p(y,z,\pi ,\bt | \al, \eta) &= \prod_{a=1}^{N} \prod_{b > a}^{N} p(y_{ab} | z_{ab}, z_{ba}, \bt) p(z_{ab}|\pi_a)\nonumber \\ &p(z_{ba}|\pi_b) \pd{a}{N}p(\pi_a|\al) \pd{k}{K} p(\bt_k|\eta).\label{eqn:joint}
\end{align}

Both variational inference [1,4,16] and collapsed Gibbs sampling algorithms [11] have been used successfully for small to medium scale problems. However, the $\cO(N^2)$ computational complexity per update prevents it from being applied to large scale networks. A stochastic variational algorithm was developed in [1] to address this issue, where each update only depends on a small mini-batch of the nodes in the network. 


%%%%%%%%%%%%%%%%%%%%%%%%%%%%%%%%%%%%%%%%%%%%%%%%%%%%%%%%%%%%%%%
\section{Stochastic Gradient MCMC Algorithms}
Our algorithm will be based on the stochastic gradient Langevin dynamics (SGLD) [3]. To sample from a posterior distribution $p(\ta|\cX) \propto p(\cX|\ta) p(\ta)$ given $N$ i.i.d. data points $\cX=\{x_i\}_{i=1}^{N}$, SGLD applies the following update rule:
\bea
\ta^* \law \ta + \f{\epsilon_t}{2}\left(\nabla_{\ta}\log p(\ta_t)+N\bar{g}(\ta;\SGLDMinibatch)\right) + \xi, \label{eqn:sgld_update}
\eea
where $\xi \sim \cN(0,\epsilon_t )$ with $\ep_t$ the step size, $\SGLDMinibatch$ a mini-batch of size $n$ sampled from $\cX$, and $\bar{g}(\ta;\SGLDMinibatch) = \frac{1}{|\SGLDMinibatch|}\sum_{x\in \SGLDMinibatch}^{}\grad_{\ta} \log p(x|\ta)$. As the step size goes to zero by a schedule satisfying $\sm{t}{\infty}\ep_t = \infty$ and $\sm{t}{\infty}\ep_t^2 < \infty$, SGLD samples from the true posterior distribution. In SGLD, the Metropolis-Hastings (MH) accept-reject tests are ignored since the rejection probability goes to zero as the step size collapses to zero. While for a finite step size this results in some bias, the overall error is reduced by the reduction of variance due to the ability to draw many more samples per unit time. 

SGLD originated from the Langevin Monte Carlo (LMC) [15] where, unlike SGLD, the gradient is computed exactly using all data points and then a Metropolis-Hastings accept-reject test is applied. Because at each iteration SGLD requires to process only a mini-batch $\SGLDMinibatch$ and ignores the MH test, the computation complexity per iteration is only $\cO(n)$ as opposed to $\cO(N)$ of LMC. Any mini-batch sampling algorithm in the form of Eqn. \eqref{eqn:sgld_update} is called valid SGLD as long as it guarantees the gradient estimator to be unbiased, i.e.,
$\eE_{\SGLDMinibatch} \left[N\bar{g}(\ta;\SGLDMinibatch)\right] = \grad_{\ta} \log p(\cX|\ta)\label{eqn:sgld_grad_estm}$ and the variance to be finite [5].

%%%%%%%%%%%%%%
%\subsection{Stochastic Gradient Riemannian Langevin Dynamics (SGRLD)}

The stochastic gradient Riemannian Langevin dynamics (SGRLD) [2] is a subclass of SGLD which is developed to sample from the probability simplex. By applying Riemannian geometry [15] and using the mini-batch estimator in Eqn. \ref{eqn:sgld_update}, it achieved state-of-the-art performance for latent Dirichlet allocation (LDA). In particular, for a $K$-dimensional probability simplex $\pi$, it uses the \textit{expanded-mean} re-parameterization trick, where the probability of a category $k$ is given by $\pi_k = \ta_k / \sum_{j=1}^K \ta_j$ with $\ta_k \sim \text{Gamma}(\al,1)$ and $\al$ a hyperparameter of the Dirichlet distribution $p(\pi | \al )$.
Then, the update rule becomes

%==============================================================

\bea
\ta_{k}^* \law \left| \ta_{k} + \f{\ep}{2} \left( \al - \ta_{k} + \f{N}{|\SGLDMinibatch|} \sum_{d \in \SGLDMinibatch} g_d(\ta_{k}) \right) + (\ta_{k})^\ha \xi_{} \right|. \label{eqn:sgrld_update}
\eea
here $g_d(\ta_{k})$ is the gradient of the log posterior w.r.t. $\ta_{k}$ on a data point $d\in \SGLDMinibatch$.



%%%%%%%%%%%%%%%%%%%%%%%%%%%%%%%%%%%%%%%%%%%%%%%%%%%%%%%%%%%%%%%
\section{Scalable MCMC for a-MMSB}
Our algorithm iterates updating local parameters $\pi$ and a global parameter $\beta$. Because both parameters lie on the probability simplex, we start from the SGRLD and modify it to be more efficient. Also, we introduce parameters $\phi$ and $\ta$ to re-parameterize $\pi$ and $\beta$ respectively. Thus, we alternatingly sample in the $\phi$ and $\ta$ spaces, and then obtain $\pi$ and $\beta$ by normalizing $\phi$ and $\ta$. From Eqn. \ref{eqn:joint}, summing over the latent variable $z$, we obtain the following joint probability,
\bea
&p(y, \pi , \bt | \al , \eta) = \prod_{a} p(\pi_a | \al ) + \prod_{a} p(\bt_k | \eta)\nn \\
&+ \prod_{a} \prod_{b > a}\sum_{z_{ab}, z_{ba}} p(y_{ab}, z_{ab}, z_{ba} | \bt , \pi_a, \pi_b)
\label{eqn:log_joint}.
\eea


%==============================================================
\subsection{Sampling the global parameter}

By the re-parameterization, we have $\beta_k=\ta_{k1}/(\ta_{k0}+\ta_{k1})$, where $\ta_{ki}\sim$ Gamma$(\eta)\propto \ta_{ki}^{\eta-1}e^{-\ta_{ki}}$. Because $p(y, \pi , \bt | \al , \eta)$ decomposes into $p(y,\bt|\pi,\eta) p(\pi|\al)$, replacing $\bt$ by $\ta$, we compute the derivative of log of Eqn. \ref{eqn:log_joint} w.r.t. $\ta_{ki}$ for $i=\{0,1\}$ as follows:
\bea
\fp{\ln p(y ,\ta | \pi, \eta )}{\ta_{ki}} =
\fp{}{\ta_{ki}} \ln p(\ta_{ki} | \eta) + \sum_{a}\sum_{b>a} g_{ab}(\ta_{ki}),
\label{eqn:grad_bt}
\eea
where $g_{ab}(\ta_{ki}) = \fp{}{\ta_{ki}} \ln \sum_{z_{ab},z_{ba}} p(y_{ab},z_{ab},z_{ba} | \ta , \pi_a, \pi_b)$ which, similar to SGRLD for LDA [2], we can rewrite as
\bea
g_{ab}(\ta_{ki}) = \eE \left[ \eI[z_{ab}=z_{ba}=k] \left( \f{|1-i-y_{ab}|}{\ta_{ki}} - \f{1}{\ta_{k}} \right)\right]. \label{eqn:G}
\eea
where $\ta_{k}=\sum_i \ta_{ki}$ and $\eI[S]$ is equal to $1$ if a condition $S$ is TRUE and $0$ otherwise. The expectation is w.r.t. the posterior distribution of latent variables $z_{ab}$ and $z_{ba}$,
\bea
&&p(z_{ab}=k,z_{ba}=l | y_{ab}, \pi_a, \pi_b, \bt )\\ &&\propto f_{ab}^{(y)}(k,l)=
\begin{cases}
\bt_k^y(1-\bt_k)^{(1-y)}\pi_{ak}\pi_{bk}, & \mbox{if } k=l\\
\dt^{y}(1-\dt)^{(1-y)} \pi_{ak} \pi_{bl}, & \mbox{if } k \neq l \nn
\end{cases}
\label{eqn:case}
\eea
here we used simple notation $y$ instead of $y_{ab}$. Unlike the SGRLD for LDA [2], we compute the expectation in Eqn. \eqref{eqn:G} analytically by computing the normalization constant $Z_{ab}^{(y)} = \sm{k}{K}\sm{l}{K} f_{ab}^{(y)}(k,l)$ which can be reduced to $\cO(K)$ computation as follows
\begin{align}
&Z_{ab}^{(y)} = \dt^{y}(1-\dt)^{(1-y)} \nonumber\\&+\sm{k}{K} \left( \bt_k^{y}(1-\bt_k)^{(1-y)} - \dt^{y}(1-\dt)^{(1-y)}\right)\pi_{ak}\pi_{bk}
\end{align}

Then Eqn. \eqref{eqn:G} becomes
\bea
g_{ab}(\ta_{ki}) 
&=& \f{f_{ab}^{(y)}(k,k)}{Z_{ab}^{(y)}} \left(\f{|1-i-y|}{\ta_{ki}} - \f{1}{\ta_k} \right).
\eea
Plugging this into Eqn. \ref{eqn:sgrld_update}, we obtain the update rule for the global parameter,

\bea
\ta_{ki}^* \law \left| \ta_{ki} + \f{\ep}{2} \left\{ \eta - \ta_{ki} + h(\Minibatcht)\sum_{(a,b) \in \Minibatcht}g_{ab}(\ta_{ki})\right\} \right.\nn \\ \left.+ (\ta_{ki})^{\ha}\xi_{ki} \right|, \label{eqn:global_update}
\eea
here $\Minibatcht$ is a mini-batch of $n_t$ node pairs sampled from $\cE^*$ for which we use the following strategy.

\textbf{Stratified sampling:} considering that the number of links is much smaller than that of non-links, we can reduce the variance of the gradient using stratified sampling, similar to the method used in [1]. For this, at every iteration we first randomly select a node $a$ and then toss a coin with probability 0.5 to decide whether to sample link edges or non-link edges for node $a$. If it is a link, we assign all of the link edges of node $a$ to $\cE_{n_t}$. Otherwise, i.e. if it is non-link, we uniformly sample a mini-batch of $N/m$ non-link edges from the entire set of non-link edges and assign it to $\cE_{n_t}$. Here, the $m$ is a hyper-parameter. Note that the size of $|\cE_{n_t}|$ will thus be much smaller than the total number of $N(N-1)/2$ edges when $m$ is reasonably large. Then, to ensure that the gradient is unbiased, a \textit{scaling parameter} $h(\cE_{n_t})$ is multiplied. Specifically, $h(\cE_{n_t})$ is set to $N$ when $\cE_{n_t}$ is a set of link edges and to $mN$ otherwise. 

Because the global parameters $\{\beta_k\}$ does not change very fast compared to the local parameters $\{\pi_{ak}\}$, in practice we update only a random subset of the $\{\beta_k\}$ at each iteration.

\subsection{Sampling the local parameters}

Similar to the global parameter, we re-parameterize the local parameter $\pi_a$ such that $\pi_{ak} = \phi_{ak} / \sum_{j=1}^{K}\phi_{aj}$, with $\phi_{ak}\sim$ Gamma$(\alpha)\propto \phi_{ak}^{\alpha-1}e^{-\phi_{ak}}$. Then, taking the derivative of the log of Eqn. \ref{eqn:log_joint} w.r.t. $\phi_{ak}$, we obtain
\bea
\fp{\ln p(y , \phi | \bt, \al)}{\phi_{ak}} = \fp{}{\phi_{ak}} \ln p(\phi_{ak} | \al) +
\sum_{b} g_{ab}(\phi_{ak})
\eea
where $g_{ab}(\phi_{ak}) = \fp{}{\phi_{ak}} \ln \sum_{z_{ab}, z_{ba}} p(y_{ab}, z_{ab}, z_{ba} | \bt , \phi_a, \phi_b)$ which can be written as 
\bea 
g_{ab}(\phi_{ak}) = \eE\left[ \f{\eI [z_{ab} = k]}{ \phi_{ak}} - \f{1}{\phi_{a\cdot}} \right].
\label{eqn:grad_local}
\eea
Here the expectation is w.r.t. the distribution in Eqn. \eqref{eqn:case}. To compute the expectation analytically, we first integrate out $z_{ba}$ from Eqn. \eqref{eqn:case} because the expectation depends only on $z_{ab}$, and obtain the following probability up to a normalization constant

\begin{align}
&f_{ab}^{(y)}(k) = \sum_{l=1}^K f_{ab}^{(y)}(k,l)\nonumber \\ =& \pi_{ak}
\left\{ \bt_k^{y}(1-\bt_k)^{(1-y)} \pi_{bk} + \dt^{y}(1-\dt)^{(1-y)} (1- \pi_{bk}) \right\}.\label{eqn:local_f}
\end{align}

Then we obtain the normalization term by $Z_{ab}^{(y)} = \sm{k}{K}f_{ab}^{(y)}(k)$. 
Integrating out the expectation in Eqn. \eqref{eqn:grad_local}, we obtain 
\bea
g_{ab}(\phi_{ak}) = \f{f_{ab}^{(y)}(k)}{Z_{ab}^{(y)}\phi_{ak}} - \f{1}{\phi_a}.
\eea 
Plugging this to Eqn. \ref{eqn:sgrld_update}, we obtain the SGRLD update rule for the local parameter $\phi_{ak}$
\bea
\phi_{ak}^* \law \left| \phi_{ak} + \f{\ep}{2} \left( \al - \phi_{ak} + \f{N}{|\Neighbors|} \sum_{b \in \Neighbors} g_{ab}(\phi_{ak})\right) \right. \nn \\ \left. + (\phi_{ak})^{\ha} \xi_{ak}\right|.\label{eqn:local_update}
\eea
Here, the $\Neighbors$ is the neighbor set for a mini-batch node, another random mini-batch of $n$ nodes sampled from $\cV^*$. Note that $|\Neighbors| \ll |\cV^*|=N$.

\end{comment}

\subsection{Related work on parallel community detection}

Among previous work on parallelizing MMSB algorithms, we mention the Online
Tensor approach~\cite{DBLP:journals/corr/HuangNHVA13} on GPUs, and our previous
parallel implementation of SG-MCMC on GPUs and multi-core CPUs (submitted for
publication). There are a few projects that did a distributed implementation
for community detection on very large network graphs. In contrast to our work,
they all detect non-overlapping communities: \cite{Bu2013246} investigate a
large number of networks; \cite{2015arXiv150302115L} investigate hierarchical
stochastic blockmodels; \cite{Prat-Perez:2014:HQS:2566486.2568010} use
multi-core machines.

\begin{comment}
In this paper, we describe our custom RDMA D-KV (Distributed Key-Value)
store. Current RMDA D-KV store implementations are RamCloud~\cite{RamCloud},
Pilaf~\cite{Pilaf}, Herd~\cite{Herd} and FaRM~\cite{FaRM}. All these systems
use RDMA to implement a D-KV store. However, all of them are far more powerful
than our custom implementation -- and this power comes at a cost that we
can avoid. They implement a generic D-KV store that controls concurrency,
supports dynamic inserts and deletes, supports variable-sized values
(whose size may change at an update), and keys of arbitrary type. Because
of the nature of our distributed algorithm, we have to deal with none of
these issues. For us, values are fixed-size, allocated only at the initial
population, and remain alive forever. We have no concurrency between writes
and reads or other writes. Our keys are a contiguous range of integers. All
these properties together allow an extremely low-overhead implementation
that does not involve the remote host in any transaction.
\end{comment}

\section{Conclusions}
\label{sec-conclusion}

Modern machine learning algorithms are empowering their
users to solve or approximate solutions to previously intractable problems.
However, these advancements come at the cost of significant computational
complexity and long running learning processes which hinder our ability to reap
such algorithms' benefits when applied to large problems. There is a clear need
to assess such algorithms and identify optimization opportunities in order to
scale their performance.

The SG-MCMC algorithm discussed in this paper posed additional computational
challenges due to its unique stochastic nature and \emph{high data intensity}. Unlike
common machine learning algorithms, its data dependencies and memory access
patterns are \emph{nondeterministic}. In this paper, we presented our methodology of
improving the algorithm's performance by restructuring it to
cater for concurrent and distributed parallelism.

We showed that the algorithm's state can be reduced by 75\% after a thorough
analysis of the computational patterns and data structures, thus significantly
reducing its data intensity.

In our implementation for many-core processors, we navigated the complex optimization
landscape by dynamically generating kernel code and investigating different
combinations of optimizations. The outcome of these efforts culminated in
significant speedup factors of 21 and 120 using a multi-core CPU and a GPU
respectively. These speedup numbers were achieved in comparison to an already
optimized sequential implementation. The evaluation of the performance
across several GPUs yielded one striking find. No single optimization strategy
works best for all GPUs: different choices for memory allocation strategy,
kernel block size and vector width were required for optimal results for
different GPUs and application instances.
%
Hence, the outcome of this study reinforces the significance of avoiding premature
optimization. In particular, the success
of common GPU optimizations depends on the device and the
problem size it is applied to.

We further created a highly scalable implementation to handle the largest existing
community graphs on a distributed cluster machine.
This design solved
several problems to achieve scalability and high performance.
We overlapped computation
with communication to hide latency.  We use a
mixture of MPI and RDMA primitives to speed up the communication between cluster
nodes.
%
Further, we conducted a thorough empirical evaluation of the system to study its strong
and weak scalability on 65 cluster nodes using large data sets.
% Additionally, we assessed the efficiency of the algorithm's resource utilization.
Finally, a
demonstration of the implementation's utility was provided by processing 6
large real-world data sets.
To the best of
our knowledge, this is the first time that the problem of deducing overlapping
communities has been learned for problems of such a large scale.

% \section{Conclusions}
\label{sec-conclusion}

Modern machine learning algorithms are empowering their
users to solve or approximate solutions to previously intractable problems.
However, these advancements come at the cost of significant computational
complexity and long running learning processes which hinder our ability to reap
such algorithms' benefits when applied to large problems. There is a clear need
to assess such algorithms and identify optimization opportunities in order to
scale their performance.

The SG-MCMC algorithm discussed in this paper posed additional computational
challenges due to its unique stochastic nature and \emph{high data intensity}. Unlike
common machine learning algorithms, its data dependencies and memory access
patterns are \emph{nondeterministic}. In this paper, we presented our methodology of
improving the algorithm's performance by restructuring it to
cater for concurrent and distributed parallelism.

We showed that the algorithm's state can be reduced by 75\% after a thorough
analysis of the computational patterns and data structures, thus significantly
reducing its data intensity.

In our implementation for many-core processors, we navigated the complex optimization
landscape by dynamically generating kernel code and investigating different
combinations of optimizations. The outcome of these efforts culminated in
significant speedup factors of 21 and 120 using a multi-core CPU and a GPU
respectively. These speedup numbers were achieved in comparison to an already
optimized sequential implementation. The evaluation of the performance
across several GPUs yielded one striking find. No single optimization strategy
works best for all GPUs: different choices for memory allocation strategy,
kernel block size and vector width were required for optimal results for
different GPUs and application instances.
%
Hence, the outcome of this study reinforces the significance of avoiding premature
optimization. In particular, the success
of common GPU optimizations depends on the device and the
problem size it is applied to.

We further created a highly scalable implementation to handle the largest existing
community graphs on a distributed cluster machine.
This design solved
several problems to achieve scalability and high performance.
We overlapped computation
with communication to hide latency.  We use a
mixture of MPI and RDMA primitives to speed up the communication between cluster
nodes.
%
Further, we conducted a thorough empirical evaluation of the system to study its strong
and weak scalability on 65 cluster nodes using large data sets.
% Additionally, we assessed the efficiency of the algorithm's resource utilization.
Finally, a
demonstration of the implementation's utility was provided by processing 6
large real-world data sets.
To the best of
our knowledge, this is the first time that the problem of deducing overlapping
communities has been learned for problems of such a large scale.

% use section* for acknowledgement
\section*{Acknowledgment}

The authors would like to thank professor Max Welling, Wenzhe Li and Sungjin
Ahn for their guidance in navigating the intricacies of machine learning.
Moreover, we would like to thank Kees Verstoep for his excellent maintenance
and administration of VU/DAS5 cluster.
%
This work was partially funded by the Netherlands Organization for Scientific
Research (NWO).


% trigger a \newpage just before the given reference
% number - used to balance the columns on the last page
% adjust value as needed - may need to be readjusted if
% the document is modified later
%\IEEEtriggeratref{8}
% The "triggered" command can be changed if desired:
%\IEEEtriggercmd{\enlargethispage{-5in}}

% references section

% can use a bibliography generated by BibTeX as a .bbl file
% BibTeX documentation can be easily obtained at:
% http://www.ctan.org/tex-archive/biblio/bibtex/contrib/doc/
% The IEEEtran BibTeX style support page is at:
% http://www.michaelshell.org/tex/ieeetran/bibtex/
%\bibliographystyle{IEEEtran}
% argument is your BibTeX string definitions and bibliography database(s)
%\bibliography{IEEEabrv,../bib/paper}
%
% <OR> manually copy in the resultant .bbl file
% set second argument of \begin to the number of references
% (used to reserve space for the reference number labels box)
\bibliographystyle{IEEEtran}
\bibliography{IEEEabrv,paper}


% that's all folks
\end{document}


