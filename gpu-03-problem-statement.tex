\section{SG-MCMC Algorithm Overview} 
\label{sec-algorithm}

In this section we describe the computational aspects of the SG-MCMC MMSB
algorithm. Moreover, we will introduce the data structures and notation that
will be used throughout this paper.
%
Kindly refer to~\cite{DBLP:journals/corr/LiAW15} for a detailed explanation of
the statistical concepts of the algorithm.

The network graph $G$ consists of the undirected edges~$E$ and has $N$ vertices.
The algorithm starts by partitioning $G$ into the \textit{training
set}, the \textit{validation set} $H$ and the test set (the latter is not used
in our implementations). $H$ and the test set
are much smaller than $G$, typically between 1\% and 10\% of the edges
in~$G$.
The number of communities $K$
is specified as a model parameter to the algorithm.

The algorithm progresses by iteratively improving the global state of the learning
problem, using the training set. There are two pairs of data structures that
hold the global state. $\theta$, a $K{\times}2$ matrix, is used to
calculate
the community strength. The community strenghth represents the probability that two members in a community share
an edge. The actual community strength
is the vector $\beta$ of length $K$. As is shown in Equation~\ref{eqn-beta}, it is the
normalized version of~$\theta_{k,2}$.
The $\pi$ matrix of dimensions $N{\times}K$ represents the probability for
each vertex in $G$ to be a member of each community. It is the normalized
equivalent of the $\phi$ matrix of dimensions $N{\times}K$, on which the
calculations occur. The
definitions of $\beta$ and $\pi$ are:
%
\begin{eqnarray}
    \label{eqn-beta}
    \beta_{k} = \frac {\theta_{k,2}} {\theta^{sum}_k} &
    	\textrm{where} & \theta^{sum}_k = \sum_{j=1}^{2} \theta_{kj} \\
    \label{eqn-pi}
    \pi_{ik} = \frac {\phi_{ik}} {\phi^{sum}_i} &
    	\textrm{where} & \phi^{sum}_i = \sum_{k=1}^{K} \phi_{ik}
\end{eqnarray}

Pseudo-code for the algorithm is presented in
Listing~\ref{algorithm-pseudo-code}.

An iteration in the algorithm consists of 6 compute stages.
We will highlight the data accessed in the stages as this
determines the opportunities for parallelism.

\begin{figure}[tb]
\begin{comment}
Stochastic: iteratively:
 - sample a mini-batch $M$ (from the graph, or not from G)
 - for each node in the mini-batch:
   sample $n$ random 'neighbors' (not from G)
   for each edge (node, neighbor), calculate its contribution to the gradient
   add noise to the gradient contributions to be able to jump out of local minima
 - sum the gradients for each node in the mini-batch
 - using the gradients, infer new value of $\phi$ for each node in the mini-batch
 - update $\pi$ accordingly
 - calculate the gradients for $\beta$ for the edges in the mini-batch, and
   update $\beta$ and $\theta$
 - every so many iterations:
    + verify the quality of $\pi$ and $\beta$ against the validation set; metric is
      perplexity
    + terminate when perplexity changes less than $\epsilon$
\end{comment}


% \begin{figure}[h]
\begin{lstlisting}[mathescape,caption=SG-MCMC MMSB algorithm,label=algorithm-pseudo-code]
iterate
   sample a mini-batch $M$
   // $M$ may be a subset of $G$, or disjoint with $G$
   for each vertex $i$ in $M$:
      sample $n$ random 'neighbors' (not from $G$)
      for each edge ($i$, neighbor)
         calculate its contribution to the gradient in $\phi$
      // to be able to jump out of local minima:
      add noise to the gradient contributions
      update $\phi[i]$ using stepsize $s$
   for each vertex $i$ in $M$:
      update $\pi[i]$ according to changed $\phi[i]$
   for the edges in $M$:
      calculate gradients in $\theta$ and update $\theta$
   for the edges in $M$:
      update $\beta$ accordingly
   every so many iterations:
      // metric is perplexity:
      verify the quality of $\pi$ and $\beta$ against the validation set
until perplexity change is less than $\epsilon$
\end{lstlisting}
% \end{figure}

\end{figure}

The first stage (line~2 in the listing) randomly draws a mini-batch $M$,
using the "stratified random node" strategy~\cite{DBLP:journals/corr/LiAW15}.
In this strategy, a coin toss is used to decide between two sample types.
%
The first sample type
chooses one random vertex $i$ and selects all of its edges to constitute the
mini-batch $M$. This sample type is referred to as \textit{link edges}. On the other
hand, the second sample type randomly draws a vertex $i$ and generates
random edges of the form ($i$, $j$) such that the edges are not in $G$. This
sample type is referred to as \textit{nonlink edges}. The set of vertices that
constitute the edges of the mini-batch $M$ are represented by $m$.

In the second stage (line~4), for each vertex $i$ in~$m$, a neighbor set of
size $n$ is randomly sampled with edges of the form $(i,j)$.

Stage~3, \textit{update\_phi} (line~6-10), calculates a gradient vector
$\nabla\phi_i$ for each vertex $i$ in the mini-batch, by iterating over
the edges~${(i,j)}$ in $i$'s neighbor set; the data that is used is
$\pi_i$, $\pi_j$,
and~$\beta$. The gradient $\nabla\phi_i$ is used to update~$\phi_i$. Random
noise is added to the update to prevent the algorithm from getting confined
to a local minimum.
Stage 4 (line~11-13), \textit{update\_pi}, updates $\pi_i$ so it remains
the normalized version of~$\phi_i$.

Stage~5, \textit{update\_theta} (line~14),
uses $\beta$ and $\pi_a$, $\pi_b$ for the edges $(a,b)$ in the mini-batch~$M$ to
calculate a
gradient vector~$\nabla\theta$. $\theta$ is updated using $\nabla\theta$, again
with the addition of noise. Stage~6, \textit{update\_beta} (line~16),
recalculates $\beta$ as the normalized
version of~$\theta_{k,2}$.

At regular intervals, the algorithm's global state is assessed by
evaluating the perplexity over the edges in the validation set $H$. The
\textit{perplexity} is a metric that represents the quality of the algorithm solution at
a given point in time. It is used to detect the algorithm's
convergence.
The
perplexity, as defined in~\cite{DBLP:journals/corr/LiAW15}, is the exponential
of the average over time of the negative
log-likelihood of meeting a link edge. In this 7th stage, $\beta$ is used,
as are $\pi_a$ and $\pi_b$ for each edge $(a,b)$ in~$H$.

The graph $G$ is queried for membership in the stages \textit{update\_phi},
\textit{update\_beta}, and \textit{perplexity}. The validation set~$H$ is
traversed in \textit{perplexity}.

