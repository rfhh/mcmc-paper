\section{Evaluation}
\label{sec-evaluation}

The following subsections discuss the performance evaluation of the algorithm
across its different stages. As the system underwent multiple design mutations,
a careful analysis is needed to identify the contributions of each change
independently.
%
First, an examination is provided detailing the construction of an efficient
sequential baseline version.
This spans
the code conversion from Python to C++ as well as the effects of applying
the localized optimizations discussed in
Section~\ref{sec-design-sequential}.
%
Second, we explore the performance benefits of parallelizing the computations
on a multi-core CPU using OpenCL.
%
Next, we assess the trade-offs associated with each of the GPU optimization
strategies and their performance effects on four GPUs spanning two chip
architecture generations.
%
Finally, we present the overall outcome of accelerating the algorithm by
displaying its fast convergence over large networks.

All experiments were conducted on the VU Amsterdam DAS5 cluster. The cluster
consists of 68 compute nodes each equipped with a dual 8-core Intel Xeon
E5-2630v3 CPU clocked at 2.40GHz, 64GB of memory and 8TB of storage.
Additionally, the cluster is fitted with a number of Nvidia GPUs including GTX
TitanX, GTX980, K40c and K20m; see Table~\ref{table-gpus} for an overview of
their properties. Table~\ref{table-snap} outlines the graph
properties of the networks used to evaluate the algorithm's performance. These
networks were obtained from the SNAP collection~\cite{snapnets}.

\begin{table}[tb]
\center\begin{tabular}{l r r c}
Network & \#Vertices & \#Edges \\
\hline
CA-HepPh        &    12,008 &    118,521 \\
com-DBLP        &   317,080 &  1,049,866 \\
% com-Youtube     & 1,134,890 &  2,987,624 \\
com-LiveJournal & 3,997,962 & 34,681,189 \\
\hline
\end{tabular}
\caption{Network graphs from the Stanford Snap collection}
\label{table-snap}
\end{table}


\subsubsection{Achieving an Efficient Baseline Sequential Version}

This subsection presents the results of transforming the algorithm implementation
from Python to C++, and then removing obvious inefficiencies, to achieve a
good baseline implementation for comparison with the parallel implementation.
%
We also investigate the performance impact of using
32-bit floating point numbers in stead of 64-bit doubles.
This hardly reduces the computation intensity for the Xeon, but it very much
affects data intensity because the data sizes are halved.
%
% It has been previously shown that stochastic learning
% algorithms do not require high precision in the presence
% of statistical approximations and the addition of random
% noise [?].

Due to the performance limitations of the Python implementation, the
experimental configuration of this section is minimal.
%
Namely, we use the CA-HepPh dataset; the number of
iterations is 1000.
%Note that these trial runs stop very far before convergence; using a small
%number of
%iterations is valid because the time spent in an iteration does not vary
%during the lifetime of the algorithm.
The selected number of communities is~1024, the mini-batch size~32, the
neighbor sample size~32.

The mini-batch sampling as described in Section~\ref{sec-background} randomly
chooses either a mini-batch of link edges, whose size is the degree of one
randomly selected vertex, or a mini-batch of nonlink edges, whose size is
specified as a model parameter. In this evaluation, we only select batches
of nonlink edges because that makes the mini-batch size, and hence the
execution times, deterministic. We separately validated that the time spent
\textit{per mini-batch vertex} for samples of link edges and nonlink edges
is fully consistent.

\begin{table}[b]
\center\begin{tabular}{l d{5.1} d{2.1} d{2.1} d{2.1}}
	&	& \multicolumn{3}{c}{C++} \\
\cline{3-5}
	& & 
            \multicolumn{1}{c}{Python} &
                \multicolumn{1}{c}{float64} &
		    \multicolumn{1}{c}{float32} \\
\multicolumn{1}{l}{Algorithm stage} &\multicolumn{1}{c}{Python} &
            \multicolumn{1}{c}{idiom} &
                \multicolumn{1}{c}{(baseline)} &
		    \multicolumn{1}{c}{baseline} \\
\hline
sample mini-batch       &      0.03 &  0.03 &  0.03 & 0.02 \\
sample neighbor sets    &      8.5  &  0.5  &  0.4  & 0.35 \\
update\_phi             &  9,014    & 52.7  &  8.9  & 4.9  \\
update\_pi              &     93.1  & 40.3  &  0.11 & 0.05 \\
update\_theta\_beta     &    339    &  2.1  &  0.76 & 0.55 \\
perplexity*             &    178    &  0.18 &  0.18 & 0.26 \\
\hline
\\[-1ex]
\end{tabular}
\caption{Performance comparison between Python, C++ that follows the Python
idioms, and our baseline C++ implementations. 1000 iterations, times in
\textrm{ms} per iteration; *perplexity time divided by 100 iterations.}
\label{table-python-comparison}
\end{table}

Table~\ref{table-python-comparison} compares performance between Python,
the C++ version labeled 'Python Idiom' with the inefficiencies inherited from
Python, then our baseline C++ version which has the inefficiencies removed,
and finally the same with 32-bit floats.
%
We present timings for the compute stages of the algorithm described in
Section~\ref{sec-background}, with the exception of
\textit{update\_theta\_beta} which combines \textit{update\_theta}
and \textit{update\_beta}.
%
The numbers are times in ms. For all stages except perplexity, the times
are per iteration.
%
Once every so many iterations (hundreds or thousands in production runs),
the perplexity calculation is invoked. The perplexity time in the table
has been divided by 100 iterations; per perplexity invocation, it is large in
comparison with the iteration stage times, but it is amortized over those
iterations.

In all implementations, \textit{update\_phi} dominates the computation.
The translation from Python into 'Python Idiom' C++ speeds up this stage
by a factor~171, even though the Python implementation uses numpy for its data
structures. Comparison of the C++ table entries for 'Python idiom' and
64-bit float baseline shows that our C++ optimizations speed up this stage
by another factor of~6. Reducing the floating-point precision to 32~bits
doubles this factor, which shows that the algorithm is very much data-dominant.
%
\textit{update\_pi} has a disproportionately large speed
gain after removing inefficiencies. This was attained by limiting the
update to $\pi$ to only those values of $\pi$ that have changed in this
iteration, in stead of the full $\pi$ array (which, incidentally, is handled
rather efficiently by numpy).
\textit{update\_theta\_beta} gave fewer opportunities for C++ optimization,
in the sampling stages there was none. In \textit{mini-batch-sampling} there
is no speedup compared to Python because it consists of a call to
a fast Python random primitive that cannot be bettered from C++.
The total performance gain between Python and the 64-bit float baseline C++
implementation exceeds a factor of~1000, and this grows to a factor of over~1500
if the floating-point precision is reduced to 32~bits. The 32-bit precision
implementation will be the baseline for the parallel performance comparison.

\begin{comment}
The introduction of a custom user-space random generator brings at most
a very small
benefit. We show it, because it is necessary for the multi-threaded
implementations described in the next section, and this measurement serves to
prove that it does not harm execution speed.
\end{comment}

\subsection{Analysis of CPU Parallelism}

\begin{comment}
  \begin{table}[t]
    \centering
    \begin{tabular}{l l c c c c}
      Comparison & Data Set        & K    & M    & n  & Iterations \\
      \hline
      Python/Baseline    & CA-HepPh        & 1024 & 32   & 32 & 1000 \\
      Parallel CPU/GPU   & com-DBLP        & 1024 & 4096 & 32 & 1000 \\
      Parallel CPU       & com-LiveJournal & 1024 & 4096 & 32 & 1000 \\
    \end{tabular}
    \caption{Experiment parameter reference.}
    \label{exp-params}
  \end{table}
\end{comment}

\begin{table}[b]	% [htb]
  \centering
  \begin{tabular}{l d{2.8} d{2.8}}
    Kernel & Time (seconds) \\
    \hline
    PPX CALC    &  0.0364737 \\
    PPX ACCUM   &  0.083 \\
    SAMPLING    &  0.535599 \\
    UPDATE\_PHI & 25.6598 \\
    UPDATE\_PI  &  0.645875 \\
    THETA SUM   &  0.0483902 \\
    GRADS PAR   &  1.92919 \\
    GRADS SUM   &  9.31122 \\
    UPDATE THETA&  0.0548013 \\
    NORM THETA  &  0.001 \\
    \hline
    TOTAL    & 38.5858 \\
  \end{tabular}
  \caption{Performance break-down of multi-core CPU version without
  vectorization. Model parameters: dataset com-DBLP; K=1024, m=4096, n=32.}
  \label{TABLE-CPU}
\end{table}

This section discusses the use of the multi-core CPU available on the DAS5
cluster. The parallel OpenCL version divides the work across the CPU cores and
performs independent calculations concurrently. As shown in
Table~\ref{TABLE-CPU}, the dominant kernel in the computation is
\textit{update\_phi},
which accounts for 66.5\% of the computation time. Without exploiting
the dual 8-core processor's vectorization capabilities, the speedup relative to
the baseline sequential C++ version is~9.8.

% CPU
\begin{figure}[htb]
\centering
\epsfig{file=plots/cpu-vector.eps, width=\columnwidth}
\caption{Performance of CPU for varying vector width.}
\label{FIG-CPU-VECTOR}
\end{figure}

In addition to applying computations in parallel, we investigated the use of
the CPU's SIMD instructions to maximize its resource utilization.
Figure~\ref{FIG-CPU-VECTOR} presents the performance obtained by vectorizing
the kernels with varying vector widths. A key aspect in this figure is the
diminishing performance benefit for higher vector widths. As the computational
performance increases, the memory throughput becomes the leading performance
bottleneck. Moreover, using 16-wide SIMD instructions gave a slight performance
penalty compared to 8-wide SIMD. The 8-wide vector version improves the speedup
relative to the baseline version from 9.8 to~20.9.

\begin{table}[b]	% [htb]
% \center\begin{tabular}{l c c c c c c c c d{3.1}}
% Device & Architecture & cores & Clock & \multicolumn{2}{c}{GFlops} & L2 Cache & Memory & Bandwidth \\ 
%        &              &       & MHz   & single & double            & KB       & GB     & (GB/s) \\ 
% \hline
% K20m        & Tesla   &  2496 &  706        & 3520 & 1170 &      &  5 & 208 \\
% K40c        & Tesla   &  2880 &  745        & 4290 & 1430 &      & 12 & 208 \\
% GTX 980     & Maxwell &  2816 & 1126 (1216) & 4612 &  144 &      &  4 & 336.5 \\
% GTX Titan-X & Maxwell &  3072 & 1000 (1075) & 6144 &  192 & 2048 & 12 & 336.5 \\
%
%
%###########################################################################################
% SPECS FROM:
% http://www.geforce.com/hardware/desktop-gpus/geforce-gtx-titan-x/specifications
% http://www.geforce.com/hardware/desktop-gpus/geforce-gtx-980/specifications
% http://www.nvidia.com/content/PDF/kepler/Tesla-K40-Active-Board-Spec-BD-06949-001_v03.pdf
% http://www.nvidia.com/content/PDF/kepler/Tesla-K20-Active-BD-06499-001-v04.pdf
%
% FLOPS DATA:
% K40: http://international.download.nvidia.com/pdf/kepler/TeslaK80-datasheet.pdf
% K20 & K40: http://www.nvidia.com/content/tesla/pdf/nvidia-tesla-kepler-family-datasheet.pdf
% GTX980 & TITANX: https://en.wikipedia.org/wiki/List_of_Nvidia_graphics_processing_units
\center\begin{tabular}{lc c c c c}
                         && \multicolumn{4} {c}{Device} \\
\cline{3-6}
Specification            && K20m  & K40c  & GTX 980 & GTX Titan-X \\
\cline{1-1}\cline{3-6}
Number of Cores          && 2496  & 2880  & 2048    & 3072  \\
Clock (MHz)              && 706   & 745   & 1126    & 1000  \\
GFlops (single)          && 3520  & 4290  & 4612    & 6144  \\
GFlops (double)          && 1170  & 1430  & 144     & 192   \\
Memory (GB)              && 5     & 12    & 4       & 12    \\
Bandwidth (GB/s)         && 208   & 288   & 224     & 336.5 \\
\cline{1-1}\cline{3-6}
\end{tabular}
\caption{Properties of the GPUs used in the evaluation}
\label{table-gpus}
\end{table}

\subsection{Analysis of GPU Parallelism}
As discussed in Section~\ref{gpu-design}, there are 10 distinct flavors of the
GPU version of the algorithm. Specifically, there is the naive strategy, the
shared strategy and 8 variations of the code generation strategy. This
section investigates the effectiveness of each flavor on Nvidia's GTX TitanX,
GTX980, K40c and K20m.

% TITANX
\begin{figure*}[t]	% [htb]
  \centering
  \epsfig{file=plots/titanx-1024-strategies-w1.eps, width=\textwidth}
  \caption{Execution time of 1000 \textit{update\_phi} invocations using the TitanX GPU,
  without explicit kernel vectorization, across a sweep of
  \textit{update\_phi} thread block
  sizes. Relevant model parameters: K=1024, M=4096, n=32.}
  \label{titanx-w1-sweep}
\end{figure*}

% TITANX K=2K
\begin{figure*}[t]	% [htb]
  \centering
  \epsfig{file=plots/titanx-2048-strategies-w1.eps, width=\textwidth}
  \caption{Execution time of 1000 \textit{update\_phi} invocations using the TitanX GPU,
  without explicit kernel vectorization, across a sweep of
  \textit{update\_phi} thread block
  sizes. Relevant model parameters: K=2048, M=4096, n=32.}
  \label{titanx-w1-sweep-2k}
\end{figure*}

Figure \ref{titanx-w1-sweep}(a) presents the performance of the GTX TitanX GPU
for all 10 strategy flavors, without explicitly vectorizing the kernels. The
x-axis represents different \textit{update\_phi} thread block sizes while the y-axis presents
the total execution time of 1000 invocations of the \textit{update\_phi} kernel. The
naive and shared strategies are labeled NAIVE and SHARED respectively. Further,
each flavor of the code generation strategy is labeled by GEN followed by the 3
choices that identify it.
%
The wide performance range inhibits the readability of the figure,
therefore, a more focused Figure~\ref{titanx-w1-sweep}(b) is provided. As would
be expected, the naive strategy exhibits the worst performance over all
thread block
sizes, as it does not explicitly cache repeated device memory read operations.
The SHARED and \textit{GEN-SSS} strategies come next in terms of performance.
Both strategies cache $Grads_i$, $Probs_i$ and $\pi_i$ in shared memory but
differ in one aspect. Namely, \textit{GEN-SSS} explicitly unrolls the internal
loops of the kernel. However, there is no significant performance difference
between them. The other flavors of the code generation strategy attain higher
performance as they unroll internal loops as well as cache data in
registers. The TitanX obtains the best performance with the \textit{GEN-RSS}
strategy and a thread block size of 64. The results of explicitly vectorized kernels
are omitted as they obtain worse performance on the TitanX.

A key model parameter that affects the behavior of the optimization strategies
is the number of communities $K$. Figure~\ref{titanx-w1-sweep-2k} presents the
same model configuration as in Figure~\ref{titanx-w1-sweep} but $K=2048$
instead of $K=1024$. One key difference between the two figures is the optimal
thread block size. An increase in $K$ comes with a proportional increase in
the size of shared memory space that is used by each thread block for the
strategies that employ shared memory. Similarly, GEN strategies that use the
register file will require additional space. Therefore, the number of
concurrent thread blocks that can execute on a single streaming multiprocessor
will decrease, minimizing the GPU's occupancy and utilization. This effect is
most clear when comparing the NAIVE with the SHARED and GEN-SSS strategies. In
this case, the NAIVE strategy outperforms both SHARED and GEN-SSS for a
thread block
size of 32 due to their low occupancy. This limitation can be counteracted by
selecting a larger thread block size which in turn increases the computation
concurrency and occupancy.

% K40
\begin{figure*}[t]	% [hbt]
  \centering
  \epsfig{file=plots/k40-1024-strategies-w2.eps, width=\textwidth}
  \caption{Execution time of 1000 \textit{update\_phi} invocations using the K40c GPU, with
  explicit kernel vectorization of width 2, across a sweep of
  \textit{update\_phi} thread block
  sizes. Relevant model parameters: K=1024, M=4096, n=32.}
  \label{k40-w2-sweep}
\end{figure*}

% \looseness=-1
Figures~\ref{k40-w2-sweep} (a) and (b) present the performance of the K40c GPU
for the same experimental configuration as shown for the TitanX in
Figure~\ref{titanx-w1-sweep} with one exception. Namely, the code version used
for this plot is explicitly vectorized with a vector width of~2. The results
for the other code versions with vector width 4 and no vectorization are
omitted as they exhibit lower performance.
%
Surprisingly, Figure~\ref{k40-w2-sweep} shows that the NAIVE strategy
outperforms SHARED and some of the GEN strategies. This can be explained by the
unique properties of the Tesla Super Computing line of products to which the
K40c belongs. These GPUs include enhanced L2 caching mechanisms that
accelerate repeated and sparse memory accesses. This is especially advantageous
as it caches repeated reads across streaming multiprocessors. However, the
highest performance is attained by \textit{GEN-RRS} which explicitly employs
registers for both $Probs_i$ and $Grads_i$.

The performance results obtained from the GTX TitanX and Tesla K40c GPUs
reinforce the importance of customizing compute kernels to each GPU's specific
architecture and capabilities. For instance, each GPU achieved its highest
performance by employing a different strategy. Moreover, each GPU displayed
different strategy-performance orderings.

\begin{figure}[b]	% [htb]
  \centering
  \epsfig{file=plots/gpus.eps, width=0.9\columnwidth}
  \caption{Speedup comparison of best performing parallel configurations of
  each compute device normalized over baseline C++ version. Relevant model
  parameters: K=1024, M=4096, n=32.}
  \label{gpus-fig}
\end{figure}

\begin{figure}[b]	% [htb]
  \centering
  \epsfig{file=plots/gpu-vector-sweep.eps, width=0.9\columnwidth}
  \caption{Performance of the best performing strategy per GPU, varying the
  vector width, normalized over the non-vectorized kernel. Relevant model
  parameters: K=1024, M=4096, n=32.}
  \label{gpus-v-sweep}
\end{figure}

\subsection{Comparison of Compute Devices}
Figure~\ref{gpus-fig} compares the highest speedup achieved by each of the
GTX TitanX, GTX980, K40c, K20m and the Intel Xeon CPU relative to the baseline
sequential C++ version. These results are consistent with the relative
capabilities of each device as listed in Table~\ref{table-gpus}. For instance,
the TitanX achieves the highest speedup of 86 relative to the baseline. At the
other end, the parallel CPU version attains a speedup factor of~20.9.

As vectorization is a popular GPU optimization, it is important to note that
both GTX GPUs achieved their highest performance with non-vectorized kernels.
On the other hand, the Tesla GPUs obtained the best performance with a vector
width of~2. Figure~\ref{gpus-v-sweep} presents the execution time of the best
performing strategy for each of the 4 GPUs. In this figure, the performance is
normalized over the non-vectorized kernel version for each GPU. For instance,
the TitanX exhibits an overhead factor of approximately 1.8 when using a vector
width of~4. On the other hand, the K40c improves its performance by roughly
20\% when it uses a vector width of~2 compared to the non-vectorized kernel.
Therefore, explicit vectorization of the kernels can be either useful or
harmful depending on the GPU architecture and the specific problem it is
applied to.





\begin{comment}
\begin{figure}[tb]	% [htb]
  \centering
  \epsfig{file=plots/ppx-gpu.eps, width=\columnwidth}
  \caption{Perplexity convergence of com-DBLP on the Maxwell \mbox{Titan-X}~GPU. Number
  of communities chosen to maximally fill the GPU memory.}
  \label{fig-ppx-gpu}
\end{figure}

\begin{figure}[tb]	% [htb]
  \centering
  \epsfig{file=plots/ppx-cpu.eps, width=\columnwidth}
  \caption{Perplexity convergence of com-LiveJournal on the multi-core~CPU.
  Number of communities chosen to maximally fill the CPU memory.}
  \label{fig-ppx-cpu}
\end{figure}
\end{comment}

\begin{figure}[tb]	% [htb]
  \centering
  \epsfig{file=plots/ppx-gpu-cpu.eps, width=0.45\textwidth}
  \caption{Perplexity convergence. (a)~com-DBLP on the Maxwell \mbox{Titan-X}~GPU.
  (b)~com-LiveJournal on the multi-core~CPU. Number of communities chosen to
  maximally fill the memory of GPU and CPU respectively. M=4K, $|\Neighbors|$=32.}
  \label{fig-ppx-gpu-cpu}
\end{figure}

\subsubsection{Algorithm Convergence}

This section presents the evolution of the perplexity, i.e.\ the metric for
the solution quality, as a function of time for
two datasets, whose properties are given in Table~\ref{table-snap}. This
analysis shows that our implementations of the algorithm indeed achieve
convergence, and it gives an impression of the time required to achieve
convergence for the largest datasets and community sizes that fit into the memory
of a GPU and a CPU respectively. The dominant customer of memory is the
$\pi$ matrix, sized $N{\times}K$ for $K$ communities and $N$ vertices in the
graph. The runtime per iteration is independent of~$N$, but it depends on the
product of~$K$, the mini-batch size~$M$, and the neighbor sample size~$|\Neighbors|$. On
the other hand, the number of iterations required to achieve sufficient
convergence does grow with the size of the graph.

Figure~\ref{fig-ppx-gpu-cpu}(a) shows the perplexity development for the dataset
com-DBLP, with $N=317,080$. It is computed on the Maxwell \mbox{Titan-X} GPU with input
parameter set $K=4096$, $m=4096$, $n=32$. Convergence is reached after 6~hours.

Figure~\ref{fig-ppx-gpu-cpu}(b) shows the perplexity convergence for
com-LiveJournal, with $N=3,997,962$, computed on a multi-core DAS5 node.
To fill the CPU core memory~(64GB), the input parameter set is $K$=3072,
$M$=4096, $|\Neighbors|$=32. Sufficient convergence is reached after 18~hours. Note
that this network graph is so large that the GTX~\mbox{Titan-X} would be unable to
process more than $K$=256 communities.

