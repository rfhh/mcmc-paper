\section{Evaluation}
\label{sec-evaluation}

The following subsections discuss the performance evaluation of the algorithm
across its different stages. As the system underwent multiple design mutations,
a careful analysis is needed to identify the contributions of each change
independently.
%
First, an examination is provided detailing the construction of an efficient
sequential baseline version.
This spans
the code conversion from Python to C++ as well as the effects of applying
the localized optimizations discussed in
Section~\ref{sec-design-sequential}.
%
Second, we explore the performance benefits of parallelizing the computations
on a multi-core CPU using OpenCL.
%
Next, we assess the trade-offs associated with each of the GPU optimization
strategies and their performance effects on four GPUs spanning two chip
architecture generations.
%
Finally, we present the overall outcome of accelerating the algorithm by
displaying its fast convergence over large networks.

All experiments were conducted on the VU Amsterdam DAS5 cluster. The cluster
consists of 68 compute nodes each equipped with a dual 8-core Intel Xeon
E5-2630v3 CPU clocked at 2.40GHz, 64GB of memory and 8TB of storage.
Additionally, the cluster is fitted with a number of Nvidia GPUs including GTX
TitanX, GTX980, K40c and K20m; see Table~\ref{table-gpus} for an overview of
their properties. Table~\ref{table-snap} outlines the graph
properties of the networks used to evaluate the algorithm's performance. These
networks were obtained from the SNAP collection~\cite{snapnets}.

\begin{table}[tb]
\center\begin{tabular}{l r r c}
Network & \#Vertices & \#Edges \\
\hline
CA-HepPh        &    12,008 &    118,521 \\
com-DBLP        &   317,080 &  1,049,866 \\
% com-Youtube     & 1,134,890 &  2,987,624 \\
com-LiveJournal & 3,997,962 & 34,681,189 \\
\hline
\end{tabular}
\caption{Network graphs from the Stanford Snap collection}
\label{table-snap}
\end{table}


\input{05-02-sequential}
\input{05-03-accelerator}
\input{05-05-perplexity}
